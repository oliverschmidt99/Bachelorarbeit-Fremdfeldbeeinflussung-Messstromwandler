
\clubpenalty = 10000 % schliesst Schusterjungen aus
\widowpenalty = 10000 % schliesst Hurenkinder aus
\hyphenpenalty = 1000 % Keine Silbentrennung

\setbeamercovered{transparent}
%\beamertemplatenavigationsymbolsempty

\useoutertheme{split}
\useinnertheme{default}
\setbeamertemplate{blocks}[rounded][shadow=false]
\usefonttheme{professionalfonts}

% ###### color definition ######
\definecolor{fbtechnik}{gray}{.2}
\definecolor{logobg}{rgb}{0.537,0.7765,0.796}
\definecolor{cHSblue}{RGB}{0,93,164}
\definecolor{cHSmint}{RGB}{138,198,203}
\definecolor{cHSfont}{RGB}{88,88,90}

% ###### color setup ######
\setbeamercolor{section in head/foot}{bg=white,fg=black}
\setbeamercolor{subsection in head/foot}{bg=cHSmint,fg=white}
\setbeamercolor{block body}{bg=white,fg=black}
\setbeamercolor{block title}{bg=white,fg=cHSblue}
\setbeamercolor{item projected}{fg=black,bg=black!20}
\setbeamercolor{item}{fg=cHSblue,bg=black!20}
\setbeamercolor{frametitle}{bg=cHSblue,fg=white}
\setbeamercolor{framesubtitle}{bg=cHSblue!25,fg=black!80}
\setbeamercolor{title}{bg=cHSblue,fg=white}
\setbeamercolor{subtitle}{bg=cHSblue,fg=white}
\setbeamercolor{logo}{fg=logobg,bg=logobg}
\setbeamercolor{title in head/foot}{bg=white,fg=black} % Für Kopfzeile
\setbeamercolor{author in head/foot}{bg=cHSblue,fg=red}
\setbeamercolor{structure}{bg=white,fg=black}
\setbeamercolor{normal text}{bg=white,fg=black}
\setbeamercolor{progress bar background}{bg=gray!30}
\setbeamercolor{progress bar foreground}{bg=cHSblue}
% --- Neue Farben für die Kopfzeile ---
\setbeamercolor{headline left}{bg=white}
\setbeamercolor{headline right}{bg=white}


% ###### font definition for head/foot ######
\setbeamerfont{title in head/foot}{size=\normalsize}
\setbeamerfont{footline content}{size=\scriptsize}

% ###### headline definition (NEU & KORRIGIERT) ######
\setbeamertemplate{headline}
{
  \leavevmode%
  \hbox{%
    % Linke Box: Logo
    \begin{beamercolorbox}[wd=0.4\paperwidth,ht=1.0cm,dp=0.5cm,left]{headline left}
      \hspace{1cm} % Linker Abstand
      % PLATZHALTER: Ersetze dies durch deinen \includegraphics Befehl
      \texttt{\includegraphics[height=1.5cm,keepaspectratio]{03_Ressourcen/Logo/logo_rolf-janssen_2024.pdf}}
    \end{beamercolorbox}%
    % Rechte Box: Titel
    \begin{beamercolorbox}[wd=0.6\paperwidth,ht=1.0cm,dp=0.5cm,right]{headline right}
      \usebeamerfont{title in head/foot}%
      \insertshorttitle%
      \hspace{1cm} % Rechter Abstand
    \end{beamercolorbox}%
  }%
  \vspace{1mm}
}

% ###### footline definition (NEU) ######
\setbeamertemplate{footline}{%
  % Text-Zeile
  \leavevmode\hbox{%
    % Linke Box: Aktueller Abschnitt
    \begin{beamercolorbox}[wd=0.7\paperwidth,ht=2.25ex,dp=1ex,leftskip=1cm,left]{section in head/foot}%
      \usebeamerfont{footline content}\insertsectionhead%
    \end{beamercolorbox}%
    % Rechte Box: Seitenzahl
    \begin{beamercolorbox}[wd=0.3\paperwidth,ht=2.25ex,dp=1ex,rightskip=1cm,right]{section in head/foot}%
      \usebeamerfont{footline content}\insertframenumber/\inserttotalframenumber%
    \end{beamercolorbox}%
  }%
  \vskip0pt%
  % Fortschrittsbalken
  \begin{beamercolorbox}[wd=\paperwidth,ht=1ex,dp=0ex]{progress bar background}%
    \ifnum\inserttotalframenumber>0%
      \begin{beamercolorbox}[wd=\dimexpr\paperwidth*\insertframenumber/\inserttotalframenumber\relax,ht=1ex,dp=0ex]{progress bar foreground}%
      \end{beamercolorbox}%
    \fi
  \end{beamercolorbox}%
}
% ###### Navigationssymbole entfernen ######
\setbeamertemplate{navigation symbols}{}

% ###### frametitle definition ######
\setbeamertemplate{frametitle}{\leavevmode\vbox{%
    \begin{beamercolorbox}[wd=1\paperwidth,ht=3ex,dp=1.125ex,leftskip=0.4cm,rightskip=.15cm plus1fil]{frametitle}
      \usebeamerfont{frametitle}%
      \vbox{} \vspace{-0.1cm}
      \strut\Large{}\insertframetitle\strut\par%
      \vskip6pt
      {%
        \ifx\insertframesubtitle\@empty%
          \else%
          {\usebeamerfont{framesubtitle}\usebeamercolor[fg]{framesubtitle}\insertframesubtitle\strut\par}%
        \fi
      }%
      \vskip-1ex%
    \end{beamercolorbox}%
  }\vskip0pt}

% ###### custom environments ######
\newenvironment{eblock}{\setbeamercolor{block body}{bg=cHSblue!40,fg=black}\begin{block}{}\vspace{-3px}}{\end{block}}
\newenvironment{wblock}{\setbeamercolor{block body}{bg=logobg!2,fg=black}\begin{block}{}\vspace{-3px}}{\end{block}}

\parindent0.0cm
\parskip1.5ex plus0.5ex minus0.5ex % KORRIGIERT: Komma durch Punkt ersetzt

% ###### custom commands (KORRIGIERT) ######
\newcommand{\titleframe}{
  % HINWEIS: \includegraphics für Hintergrund wurde durch Platzhalter ersetzt.
  % \setbeamertemplate{background canvas}{\includegraphics[width=1\paperwidth]{hintergrund}}
  \begin{frame}[plain]
    \begin{center}
      % PLATZHALTER: Ersetze dies durch deine \includegraphics Befehle
      \texttt{\includegraphics[height=1.5cm,keepaspectratio]{03_Ressourcen/Logo/logo_rolf-janssen_2024.pdf}}\quad
      \texttt{\includegraphics[height=1.5cm,keepaspectratio]{03_Ressourcen/Logo/logo.pdf}}
      \color{fbtechnik}
    \end{center}
    \maketitle
  \end{frame}
  \setbeamertemplate{background canvas}{}
}