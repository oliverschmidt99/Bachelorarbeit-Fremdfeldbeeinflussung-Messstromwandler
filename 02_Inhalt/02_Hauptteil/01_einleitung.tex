% =============================================================================
% KAPITEL 1: EINLEITUNG (01_einleitung.tex)
% -----------------------------------------------------------------------------
% Hier führst du den Leser in das Thema ein.
% Du stellst die Relevanz
% dar, formulierst die Forschungsfrage und gibst einen Überblick über
% den Aufbau der Arbeit.
% =============================================================================

\section{Einleitung}
\label{chap:einleitung}

In Niederspannungsschaltanlagen werden immer kompakter gebaut, sodass man in einem Schaltraum möglichst viele Ein und Ausgänge schalten, messen und Anbinden kann, durch die kompakte Bauform werden die Messungen verfälscht durch die enge Baufrom und den hohen Strömen, fangen die Messstromwandler

\subsection{Motivation}
\label{sec:motivation}

\subsection{Problemstellung}
\label{sec:problemstellung}

\subsection{Zielsetzung}
\label{sec:zielsetzung}

\subsection{Vorgehensweise}
\label{sec:vorgehensweise}