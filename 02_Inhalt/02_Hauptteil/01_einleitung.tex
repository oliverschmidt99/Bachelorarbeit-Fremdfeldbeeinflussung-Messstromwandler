% =============================================================================
% KAPITEL 1: EINLEITUNG (01_einleitung.tex)
% -----------------------------------------------------------------------------
% Hier führst du den Leser in das Thema ein.
% Du stellst die Relevanz
% dar, formulierst die Forschungsfrage und gibst einen Überblick über
% den Aufbau der Arbeit.
% =============================================================================

\section{Einleitung}
\label{chap:einleitung}



\subsection{Motivation}
\label{sec:motivation}

\subsection{Problemstellung} \label{sec:problemstellung}

\subsection{Problemstellung} \label{sec:problemstellung}

An moderne Niederspannungsschaltanlagen wird die Anforderung gestellt, eine hohe Stromtragfähigkeit bei gleichzeitig möglichst kompakter Bauweise zu gewährleisten. Aufgrund der daraus resultierenden räumlichen Nähe der Leiter zueinander und der hohen Betriebsströme sind Messstromwandler verstärkt magnetischen Fremdfeldern ausgesetzt. Physikalisch ist dies darin begründet, dass die magnetische Feldstärke mit abnehmendem Abstand zur Quelle ansteigt.

Diese Fremdfeldeinflüsse wirken sich auf die Zuverlässigkeit der Messung aus. Bereits ab einem Nennstrom von \SI{2500}{\ampere} ist eine Beeinflussung der Messwerte festzustellen. Diese Abweichung nimmt mit steigender Stromstärke zu, sodass bei Nennströmen im Bereich von \SI{5000}{\ampere} bis \SI{6000}{\ampere} betragsmäßige Messabweichungen von bis zu \SI{400}{\ampere} auftreten. Eine derartige Ungenauigkeit ist relevant, da der Markt zunehmend präzisere Messwerte für Schutzeinrichtungen, die Anlagenüberwachung und insbesondere für Verrechnungszwecke fordert. Ist die Messung durch Fremdfelder verfälscht, können Schutzorgane fehlauslösen oder Energiekosten fehlerhaft abgerechnet werden.

Zwar existieren am Markt Lösungsansätze wie kompensierte Wandler oder zusätzliche Schirmungen (z.B.\ durch ferromagnetische Bleche), doch sind diese Maßnahmen oft mit Nachteilen verbunden. Kompensierte Wandler sind in der Regel kostenintensiver und benötigen durch zusätzliche Wicklungen mehr Bauraum, der in kompakten Anlagen oft nicht vorhanden ist. Zudem ist bislang nicht hinreichend geklärt, ob diese teureren Spezialwandler bei sehr hohen Strömen tatsächlich die notwendige Prozesssicherheit bieten oder ob wirtschaftlichere Standardwandler unter bestimmten Bedingungen vergleichbare Ergebnisse liefern.

\subsection{Zielsetzung der Arbeit} \label{sec:zielsetzung}

Vor dem Hintergrund der Neuentwicklung einer Schaltanlagengeneration ist es das Ziel dieser Arbeit, eine technisch zuverlässige und zugleich wirtschaftliche Lösung für die Strommessung unter Fremdfeldeinfluss zu identifizieren.

Dazu soll systematisch untersucht werden, wie sich verschiedene Wandlertypen – von Standardausführungen bis hin zu herstellerspezifischen Speziallösungen – bei hohen Strömen verhalten. Ein zentraler Fokus liegt dabei auf der Überprüfung, ob der Einsatz kostenintensiver kompensierter Wandler in diesem Hochstrombereich erforderlich ist. Darüber hinaus wird als alternativer Lösungsansatz untersucht, ob eine Anpassung der Leitergeometrie innerhalb der Schaltanlage den Fremdfeldeinfluss konstruktiv minimieren kann. Die Ergebnisse sollen direkt in den Entwicklungsprozess der neuen Schaltanlage einfließen, um die Messgenauigkeit und Betriebssicherheit zu steigern.


\subsection{Vorgehensweise}
\label{sec:vorgehensweise}