% =============================================================================
% KAPITEL 4: AUSWERTUNG UND DISKUSSION
% =============================================================================

\section{Auswertung und Diskussion}
\label{chap:auswertung_diskussion}

In diesem Kapitel werden die Auswirkungen unterschiedlicher Leitergeometrien auf die Messgenauigkeit der Stromwandler untersucht. Dabei wird zunächst die allgemeine Beeinflussung betrachtet, bevor eine detaillierte Analyse der einzelnen Außenleiter und eine ökonomische Bewertung erfolgt.

\subsection{Einfluss der Leitergeometrie auf die Messgenauigkeit}
\label{sec:vergleich_geometrie}

Der Vergleich zwischen der Parallel- und Dreiecksanordnung zeigt die grundsätzliche Empfindlichkeit der Wandler gegenüber Fremdfeldern.

\begin{diagram}[H]
    \centering
    \includegraphics[width=1.0\textwidth]{03_Ressourcen/diagramme/dia_2000A_dreieck_parallel/dia_2000A_dreieck_parallel-Zusammenfassung_MultiCurrent.pdf}
    \caption{Zusammenfassender Vergleich der Leitergeometrien bei 2000\,A}
    \label{dia:2000A_zusammenfassung}
\end{diagram}

\begin{diagram}[H]
    \centering
    \includegraphics[width=1.0\textwidth]{03_Ressourcen/diagramme/dia_5000A_dreieck_parallel/dia_5000A_dreieck_parallel-Zusammenfassung_MultiCurrent.pdf}
    \caption{Zusammenfassender Vergleich der Leitergeometrien bei 5000\,A}
    \label{dia:5000A_zusammenfassung}
\end{diagram}

\subsection{Detaillierte Analyse der Phasenbeeinflussung}
\label{sec:detail_analyse_phasen}

Um die lokale Einwirkung der Magnetfelder besser zu verstehen, wird die Messabweichung für die einzelnen Außenleiter betrachtet. Besonders der mittlere Leiter (L2) unterliegt bei der Parallelanordnung einer starken Beeinflussung.

\begin{diagram}[H]
    \centering
    \includegraphics[width=0.9\textwidth]{03_Ressourcen/diagramme/dia_5000A_dreieck_parallel/dia_5000A_dreieck_parallel-Detail_L1_MultiCurrent.pdf}
    \caption{Messabweichung am Außenleiter L1 bei 5000\,A}
    \label{dia:5000A_detail_L1}
\end{diagram}

\begin{diagram}[H]
    \centering
    \includegraphics[width=0.9\textwidth]{03_Ressourcen/diagramme/dia_5000A_dreieck_parallel/dia_5000A_dreieck_parallel-Detail_L2_MultiCurrent.pdf}
    \caption{Messabweichung am Außenleiter L2 bei 5000\,A}
    \label{dia:5000A_detail_L2}
\end{diagram}

\begin{diagram}[H]
    \centering
    \includegraphics[width=0.9\textwidth]{03_Ressourcen/diagramme/dia_5000A_dreieck_parallel/dia_5000A_dreieck_parallel-Detail_L3_MultiCurrent.pdf}
    \caption{Messabweichung am Außenleiter L3 bei 5000\,A}
    \label{dia:5000A_detail_L3}
\end{diagram}

\subsection{Ökonomische Evaluation und Technologie-Ranking}
\label{sec:oekonomie_ranking}

In diesem Abschnitt werden die technischen Ergebnisse mit den Kosten der Wandler korreliert.

\begin{diagram}[H]
    \centering
    \includegraphics[width=0.9\textwidth]{03_Ressourcen/diagramme/dia_2000A_dreieck_parallel/dia_2000A_dreieck_parallel-Oekonomie_Ranking.pdf}
    \caption{Wirtschaftliches Ranking der Wandlertechnologien bei 2000\,A}
    \label{dia:2000A_ranking_plot}
\end{diagram}

\begin{table}[H]
    \centering
    \caption{Tabellarisches Ranking der Wandlertechnologien (2000\,A)}
    \label{tab:2000A_ranking}
    \begin{table}[H]
    \centering
    \caption{2000 A | Performance-Index | Ranking}
    \label{tab:dia_2000A_kosten_ranking}
    \begin{tabular}{p{6cm}ccccc}
        \toprule
        \textbf{Messsystem} & \textbf{\shortstack[c]{Fehler \\ Niederstrom}} & \textbf{\shortstack[c]{Fehler \\ Nennstrom}} & \textbf{\shortstack[c]{Fehler \\ Überlast}} & \textbf{\shortstack[c]{Preis \\ (€)}} & \textbf{\shortstack[c]{Fehler-Score \\ {[\%]}}} \\
        \midrule
        Celsa ALO 10030 | 2000 A \newline 1,35 $\Omega$ | Dreieck & 90,50 & 22,17 & 13,07 & 32,06 & 157,80 \\
        Celsa ALO 8030 K | 2000 A \newline 8,1 $\Omega$ | Parallel & 55,03 & 28,58 & 10,00 & 67,62 & 161,23 \\
        Celsa ALO 8030 K | 2000 A \newline 8,1 $\Omega$ | Dreieck & 59,97 & 29,13 & 9,33 & 67,62 & 166,06 \\
        Redur 13A1030.3ffp | 2000 A \newline 8,1 $\Omega$ | Parallel & 100,00 & 4,45 & 6,20 & 70,47 & 181,12 \\
        MBS ASK101.4 | 2000 A \newline 8,1 $\Omega$ | Dreieck & 52,97 & 22,61 & 7,05 & 100,00 & 182,62 \\
        MBS ASK101.4 | 2000 A \newline 8,1 $\Omega$ | Parallel & 55,06 & 28,58 & 10,02 & 100,00 & 193,65 \\
        Redur 13A1030.3ffp | 2000 A \newline 8,1 $\Omega$ | Dreieck & 97,51 & 11,81 & 17,28 & 70,47 & 197,07 \\
        Celsa ALO 10030 | 2000 A \newline 1,35 $\Omega$ | Parallel & 62,49 & 100,00 & 100,00 & 32,06 & 294,55 \\
        \bottomrule
    \end{tabular}
\end{table}
\end{table}

\begin{diagram}[H]
    \centering
    \includegraphics[width=0.9\textwidth]{03_Ressourcen/diagramme/dia_2000A_dreieck_parallel/dia_2000A_dreieck_parallel-Oekonomie_Pareto.pdf}
    \caption{Pareto-Optimierung: Messgenauigkeit vs. Anschaffungskosten bei 2000\,A}
    \label{dia:2000A_pareto}
\end{diagram}

\begin{diagram}[H]
    \centering
    \includegraphics[width=0.9\textwidth]{03_Ressourcen/diagramme/dia_5000A_dreieck_parallel/dia_5000A_dreieck_parallel-Oekonomie_Pareto.pdf}
    \caption{Pareto-Optimierung: Messgenauigkeit vs. Anschaffungskosten bei 5000\,A}
    \label{dia:5000A_pareto}
\end{diagram}