% =============================================================================
% KAPITEL 4: AUSWERTUNG UND DISKUSSION
% =============================================================================

\section{Auswertung und Diskussion}
\label{chap:auswertung_diskussion}

In diesem Kapitel werden die Auswirkungen unterschiedlicher Leitergeometrien auf die Messgenauigkeit der Stromwandler untersucht. Dabei wird zunächst die allgemeine Beeinflussung betrachtet, bevor eine detaillierte Analyse der einzelnen Außenleiter und eine ökonomische Bewertung erfolgt.

\subsection{Einfluss der Leitergeometrie auf die Messgenauigkeit}
\label{sec:vergleich_geometrie}

Der Vergleich zwischen der Parallel- und Dreiecksanordnung zeigt die grundsätzliche Empfindlichkeit der Wandler gegenüber Fremdfeldern bei verschiedenen Stromstärken.

% --- 2000 A ---
\begin{diagram}[H]
    \centering
    \includegraphics[width=1.0\textwidth]{03_Ressourcen/diagramme/dia_2000A_dreieck_parallel/dia_2000A_dreieck_parallel-Zusammenfassung_MultiCurrent.pdf}
    \caption{Zusammenfassender Vergleich der Leitergeometrien bei 2000\,A}
    \label{dia:2000A_zusammenfassung}
\end{diagram}

% --- 2500 A ---
\begin{diagram}[H]
    \centering
    \includegraphics[width=1.0\textwidth]{03_Ressourcen/diagramme/dia_2500A_dreieck_parallel/dia_2500A_dreieck_parallel-Zusammenfassung_MultiCurrent.pdf}
    \caption{Zusammenfassender Vergleich der Leitergeometrien bei 2500\,A}
    \label{dia:2500A_zusammenfassung}
\end{diagram}

% --- 3000 A ---
\begin{diagram}[H]
    \centering
    \includegraphics[width=1.0\textwidth]{03_Ressourcen/diagramme/dia_3000A_dreieck_parallel/dia_3000A_dreieck_parallel-Zusammenfassung_MultiCurrent.pdf}
    \caption{Zusammenfassender Vergleich der Leitergeometrien bei 3000\,A}
    \label{dia:3000A_zusammenfassung}
\end{diagram}

% --- 4000 A ---
\begin{diagram}[H]
    \centering
    \includegraphics[width=1.0\textwidth]{03_Ressourcen/diagramme/dia_4000A_dreieck_parallel/dia_4000A_dreieck_parallel-Zusammenfassung_MultiCurrent.pdf}
    \caption{Zusammenfassender Vergleich der Leitergeometrien bei 4000\,A}
    \label{dia:4000A_zusammenfassung}
\end{diagram}

% --- 5000 A ---
\begin{diagram}[H]
    \centering
    \includegraphics[width=1.0\textwidth]{03_Ressourcen/diagramme/dia_5000A_dreieck_parallel/dia_5000A_dreieck_parallel-Zusammenfassung_MultiCurrent.pdf}
    \caption{Zusammenfassender Vergleich der Leitergeometrien bei 5000\,A}
    \label{dia:5000A_zusammenfassung}
\end{diagram}






\subsection{Ökonomische Evaluation und Technologie-Ranking}
\label{sec:oekonomie_ranking}

In diesem Abschnitt werden die technischen Ergebnisse mit den Kosten der Wandler korreliert.

% === 2000 A ===
\begin{diagram}[H]
    \centering
    \includegraphics[width=0.9\textwidth]{03_Ressourcen/diagramme/dia_2000A_dreieck_parallel/dia_2000A_dreieck_parallel-Oekonomie_Ranking.pdf}
    \caption{Wirtschaftliches Ranking der Wandlertechnologien bei 2000\,A}
    \label{dia:2000A_ranking_plot}
\end{diagram}

\begin{table}[H]
    \centering
    \caption{Tabellarisches Ranking der Wandlertechnologien (2000\,A)}
    \label{tab:2000A_ranking}
    \begin{table}[H]
    \centering
    \caption{2000 A | Performance-Index | Ranking}
    \label{tab:dia_2000A_kosten_ranking}
    \begin{tabular}{p{6cm}ccccc}
        \toprule
        \textbf{Messsystem} & \textbf{\shortstack[c]{Fehler \\ Niederstrom}} & \textbf{\shortstack[c]{Fehler \\ Nennstrom}} & \textbf{\shortstack[c]{Fehler \\ Überlast}} & \textbf{\shortstack[c]{Preis \\ (€)}} & \textbf{\shortstack[c]{Fehler-Score \\ {[\%]}}} \\
        \midrule
        Celsa ALO 10030 | 2000 A \newline 1,35 $\Omega$ | Dreieck & 90,50 & 22,17 & 13,07 & 32,06 & 157,80 \\
        Celsa ALO 8030 K | 2000 A \newline 8,1 $\Omega$ | Parallel & 55,03 & 28,58 & 10,00 & 67,62 & 161,23 \\
        Celsa ALO 8030 K | 2000 A \newline 8,1 $\Omega$ | Dreieck & 59,97 & 29,13 & 9,33 & 67,62 & 166,06 \\
        Redur 13A1030.3ffp | 2000 A \newline 8,1 $\Omega$ | Parallel & 100,00 & 4,45 & 6,20 & 70,47 & 181,12 \\
        MBS ASK101.4 | 2000 A \newline 8,1 $\Omega$ | Dreieck & 52,97 & 22,61 & 7,05 & 100,00 & 182,62 \\
        MBS ASK101.4 | 2000 A \newline 8,1 $\Omega$ | Parallel & 55,06 & 28,58 & 10,02 & 100,00 & 193,65 \\
        Redur 13A1030.3ffp | 2000 A \newline 8,1 $\Omega$ | Dreieck & 97,51 & 11,81 & 17,28 & 70,47 & 197,07 \\
        Celsa ALO 10030 | 2000 A \newline 1,35 $\Omega$ | Parallel & 62,49 & 100,00 & 100,00 & 32,06 & 294,55 \\
        \bottomrule
    \end{tabular}
\end{table}
\end{table}


% Sonderfall Kosten/Volumen 2000A
\begin{diagram}[H]
    \centering
    \includegraphics[width=0.9\textwidth]{03_Ressourcen/diagramme/dia_2000A_kosten_volumen/dia_2000A_kosten_volumen-Oekonomie_Ranking.pdf}
    \caption{Wirtschaftliches Ranking unter Berücksichtigung des Volumens bei 2000\,A}
    \label{dia:2000A_ranking_volumen_plot}
\end{diagram}

\begin{table}[H]
    \centering
    \caption{Tabellarisches Ranking nach Kosten/Volumen (2000\,A)}
    \label{tab:2000A_ranking_volumen}
    \begin{table}[H]
    \centering
    \caption{2000 A | Performance-Index | Ranking}
    \label{tab:dia_2000A_kosten_volumen_ranking}
    \begin{tabular}{lcccccc}
        \toprule
        \textbf{Messsystem} & \textbf{Fehler Niederstrom [\%]} & \textbf{Fehler Nennstrom [\%]} & \textbf{Fehler Überlast [\%]} & \textbf{Preis (€) [\%]} & \textbf{Volumen (Gesamt) [\%]} & \textbf{Fehler-Score [\%]} \\
        \midrule
        Celsa ALO 10030 | 2000 A | 1,35 $\Omega$ | Dreieck & 90,50 & 22,17 & 13,07 & 32,06 & 81,45 & 239,26 \\
        Celsa ALO 8030 K | 2000 A | 8,1 $\Omega$ | Parallel & 55,03 & 28,58 & 10,00 & 67,62 & 91,20 & 252,43 \\
        Celsa ALO 8030 K | 2000 A | 8,1 $\Omega$ | Dreieck & 59,97 & 29,13 & 9,33 & 67,62 & 91,20 & 257,26 \\
        MBS ASK101.4 | 2000 A | 8,1 $\Omega$ | Dreieck & 52,97 & 22,61 & 7,05 & 100,00 & 94,15 & 276,78 \\
        Redur 13A1030.3ffp | 2000 A | 8,1 $\Omega$ | Parallel & 100,00 & 4,45 & 6,20 & 70,47 & 100,00 & 281,12 \\
        MBS ASK101.4 | PAC2 | 8,1 $\Omega$ | Parallel & 55,06 & 28,58 & 10,02 & 100,00 & 94,15 & 287,81 \\
        Redur 13A1030.3ffp | 2000 A | 8,1 $\Omega$ | Dreieck & 97,51 & 11,81 & 17,28 & 70,47 & 100,00 & 297,07 \\
        Celsa ALO 10030 | 2000 A | 1,35 $\Omega$ | Parallel & 62,49 & 100,00 & 100,00 & 32,06 & 81,45 & 376,00 \\
        \bottomrule
    \end{tabular}
\end{table}
\end{table}


% === 2500 A ===
\begin{diagram}[H]
    \centering
    \includegraphics[width=0.9\textwidth]{03_Ressourcen/diagramme/dia_2500A_dreieck_parallel/dia_2500A_dreieck_parallel-Oekonomie_Ranking.pdf}
    \caption{Wirtschaftliches Ranking der Wandlertechnologien bei 2500\,A}
    \label{dia:2500A_ranking_plot}
\end{diagram}

\begin{table}[H]
    \centering
    \caption{Tabellarisches Ranking der Wandlertechnologien (2500\,A)}
    \label{tab:2500A_ranking}
    \begin{table}[H]
    \centering
    \caption{2500 A | Performance-Index | Ranking}
    \label{tab:dia_2500A_kosten_ranking}
    \begin{tabular}{lccccc}
        \toprule
        \textbf{Messsystem} & \textbf{Fehler Nennstrom [\%]} & \textbf{Fehler Niederstrom [\%]} & \textbf{Fehler Überlast [\%]} & \textbf{Preis (€) [\%]} & \textbf{Fehler-Score [\%]} \\
        \midrule
        Celsa ALO 10050 K | 2500 A | 2,8 $\Omega$ | Dreieck & 2,22 & 26,04 & 2,31 & 100,00 & 130,57 \\
        Celsa ALO 10030 | 2500 A | 1,35 $\Omega$ | Dreieck & 23,06 & 40,36 & 30,92 & 43,67 & 138,01 \\
        Celsa ALO 10050 K | 2500 A | 2,8 $\Omega$ | Parallel  & 2,46 & 34,58 & 3,16 & 100,00 & 140,20 \\
        Celsa ALO 10030 | 2500 A | 1,35 $\Omega$ | Parallel & 100,00 & 100,00 & 100,00 & 43,67 & 343,67 \\
        \bottomrule
    \end{tabular}
\end{table}
\end{table}



% === 3000 A ===
\begin{diagram}[H]
    \centering
    \includegraphics[width=0.9\textwidth]{03_Ressourcen/diagramme/dia_3000A_dreieck_parallel/dia_3000A_dreieck_parallel-Oekonomie_Ranking.pdf}
    \caption{Wirtschaftliches Ranking der Wandlertechnologien bei 3000\,A}
    \label{dia:3000A_ranking_plot}
\end{diagram}

\begin{table}[H]
    \centering
    \caption{Tabellarisches Ranking der Wandlertechnologien (3000\,A)}
    \label{tab:3000A_ranking}
    \begin{table}[H]
    \centering
    \caption{3000 A | Performance-Index | Ranking}
    \label{tab:dia_3000A_kosten_ranking}
    \begin{tabular}{p{6cm}ccccc}
        \toprule
        \textbf{Messsystem} & \textbf{\shortstack[c]{Fehler \\ Nennstrom}} & \textbf{\shortstack[c]{Fehler \\ Überlast}} & \textbf{\shortstack[c]{Preis \\ (€)}} & \textbf{\shortstack[c]{Fehler \\ Niederstrom}} & \textbf{\shortstack[c]{Fehler-Score \\ {[\%]}}} \\
        \midrule
        Celsa ALO ALO 12070 | 3000 A \newline 1,35 $\Omega$ | Dreieck & 25,99 & 12,64 & 20,66 & 92,32 & 151,61 \\
        Celsa ALO 12070 K | 3000 A \newline 2,8 $\Omega$ | Parallel & 11,63 & 14,73 & 100,00 & 75,93 & 202,28 \\
        Celsa ALO 12070 K | 3000 A \newline 2,8 $\Omega$ | Dreieck & 13,60 & 2,98 & 100,00 & 97,23 & 213,81 \\
        Celsa ALO ALO 12070 | 3000 A \newline 1,35 $\Omega$ | Parallel & 100,00 & 100,00 & 20,66 & 100,00 & 320,66 \\
        \bottomrule
    \end{tabular}
\end{table}
\end{table}


% === 4000 A ===
\begin{diagram}[H]
    \centering
    \includegraphics[width=0.9\textwidth]{03_Ressourcen/diagramme/dia_4000A_dreieck_parallel/dia_4000A_dreieck_parallel-Oekonomie_Ranking.pdf}
    \caption{Wirtschaftliches Ranking der Wandlertechnologien bei 4000\,A}
    \label{dia:4000A_ranking_plot}
\end{diagram}

\begin{table}[H]
    \centering
    \caption{Tabellarisches Ranking der Wandlertechnologien (4000\,A)}
    \label{tab:4000A_ranking}
    \begin{table}[H]
    \centering
    \caption{4000 A | Performance-Index | Ranking}
    \label{tab:dia_4000A_kosten_ranking}
    \begin{tabular}{lccccc}
        \toprule
        \textbf{Messsystem} & \textbf{Fehler Nennstrom [\%]} & \textbf{Fehler Überlast [\%]} & \textbf{Fehler Niederstrom [\%]} & \textbf{Preis (€) [\%]} & \textbf{Fehler-Score [\%]} \\
        \midrule
        Celsa ALO ALO 12070 | 3000 A | 0,0 $\Omega$ | Dreieck & 18,43 & 27,07 & 89,60 & 17,79 & 152,89 \\
        Celsa ALO 12070 K | 3000 A | 0,0 $\Omega$ | Parallel  & 9,64 & 5,06 & 89,00 & 100,00 & 203,70 \\
        Celsa ALO 12070 K | 4000 A | 10,8 $\Omega$ | Dreieck  & 10,11 & 6,11 & 100,00 & 100,00 & 216,22 \\
        Celsa ALO ALO 12070 | 3000 A | 0,0 $\Omega$ | Parallel & 100,00 & 100,00 & 85,77 & 17,79 & 303,56 \\
        \bottomrule
    \end{tabular}
\end{table}
\end{table}


% === 5000 A ===
\begin{diagram}[H]
    \centering
    \includegraphics[width=0.9\textwidth]{03_Ressourcen/diagramme/dia_5000A_dreieck_parallel/dia_5000A_dreieck_parallel-Oekonomie_Ranking.pdf}
    \caption{Wirtschaftliches Ranking der Wandlertechnologien bei 5000\,A}
    \label{dia:5000A_ranking_plot}
\end{diagram}

\begin{table}[H]
    \centering
    \caption{Tabellarisches Ranking der Wandlertechnologien (5000\,A)}
    \label{tab:5000A_ranking}
    \begin{table}[H]
    \centering
    \caption{5000 A | Performance-Index | Ranking}
    \label{tab:dia_5000A_kosten_ranking}
    \begin{tabular}{p{6cm}ccccc}
        \toprule
        \textbf{Messsystem} & \textbf{\shortstack[c]{Fehler \\ Nennstrom}} & \textbf{\shortstack[c]{Fehler \\ Überlast}} & \textbf{\shortstack[c]{Fehler \\ Niederstrom}} & \textbf{\shortstack[c]{Preis \\ (€)}} & \textbf{\shortstack[c]{Fehler-Score \\ {[\%]}}} \\
        \midrule
        Celsa ALO 20060 | 5000 A \newline 0,0 $\Omega$ | Dreieck & 12,67 & 5,56 & 80,02 & 42,06 & 140,31 \\
        Celsa ALO 20060 | 5000 A \newline 0,0 $\Omega$ | Parallel & 59,44 & 56,56 & 81,37 & 42,06 & 239,43 \\
        Redur 20A1456.5FFP | 5000 A \newline 8,1 $\Omega$ | Dreieck & 36,74 & 49,56 & 75,99 & 100,00 & 262,29 \\
        MBS ASK129.10 | 5000 A \newline 10,8 $\Omega$ | Dreieck & 24,90 & 43,63 & 100,00 & 97,69 & 266,23 \\
        Redur 20A1456.5FFP | 5000 A \newline 0,0 $\Omega$ | Parallel & 100,00 & 100,00 & 89,16 & 100,00 & 389,16 \\
        \bottomrule
    \end{tabular}
\end{table}
\end{table}



% Sonderfall Kosten/Volumen 5000A
\begin{diagram}[H]
    \centering
    \includegraphics[width=0.9\textwidth]{03_Ressourcen/diagramme/dia_5000A_kosten_volumen/dia_5000A_kosten_volumen-Oekonomie_Ranking.pdf}
    \caption{Wirtschaftliches Ranking unter Berücksichtigung des Volumens bei 5000\,A}
    \label{dia:5000A_ranking_volumen_plot}
\end{diagram}

\begin{table}[H]
    \centering
    \caption{Tabellarisches Ranking nach Kosten/Volumen (5000\,A)}
    \label{tab:5000A_ranking_volumen}
    \begin{table}[H]
    \centering
    \caption{5000 A | Performance-Index | Ranking}
    \label{tab:dia_5000A_kosten_volumen_ranking}
    \begin{tabular}{lcccccc}
        \toprule
        \textbf{Messsystem} & \textbf{Fehler Nennstrom [\%]} & \textbf{Fehler Überlast [\%]} & \textbf{Fehler Niederstrom [\%]} & \textbf{Preis (€) [\%]} & \textbf{Volumen (Gesamt) [\%]} & \textbf{Fehler-Score [\%]} \\
        \midrule
        Celsa ALO 20060 | 5000 A | 0,0 $\Omega$ | Dreieck & 5,79 & 5,56 & 3,94 & 30,09 & 32,21 & 77,60 \\
        Celsa ALO 20060 | 5000 A | 0,0 $\Omega$ | Parallel & 27,17 & 56,56 & 4,01 & 30,09 & 32,21 & 150,05 \\
        MBS ASK129.10 | 5000 A | 10,8 $\Omega$ | Dreieck & 11,38 & 43,63 & 4,92 & 73,11 & 100,00 & 233,06 \\
        Celsa ALO E 16050 K | 5000 A | 0,0 $\Omega$ | Dreieck & 27,76 & 87,23 & 15,51 & 100,00 & 30,82 & 261,32 \\
        Celsa ALO E 16050 K | 5000 A | 0,0 $\Omega$ | Parallel & 100,00 & 79,40 & 100,00 & 100,00 & 30,82 & 410,21 \\
        Redur 20A1456.5vffp | 5000 A | 8,1 $\Omega$ | Dreieck & 16,80 & 49,56 & 3,74 & nan & 25,91 & nan \\
        Redur 20A1456.5vffp | 5000 A | 0,0 $\Omega$ | Parallel & 45,72 & 100,00 & 4,39 & nan & 25,91 & nan \\
        \bottomrule
    \end{tabular}
\end{table}
\end{table}


