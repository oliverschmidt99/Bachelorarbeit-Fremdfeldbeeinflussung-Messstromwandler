% =============================================================================
% KAPITEL 4: ERGEBNISSE (04_ergebnisteil.tex)
% -----------------------------------------------------------------------------
% In diesem Teil präsentierst du neutral und sachlich die Ergebnisse
% deiner Untersuchung (z.B. Messdaten, Diagramme, Beobachtungen).
% Eine Interpretation findet hier noch nicht statt.
% =============================================================================

\section{Auswertung und Diskussion}
\label{chap:auswertung_diskussion}

\subsection{Durchführung der Messreihen am optimierten Prüfstand}
\label{sec:durchfuehrung_messreihen}



\begin{diagram}[H]
    \centering
    \includegraphics[width=1.0\textwidth]{03_Ressourcen/diagramme/dia_messstrecke_neu/dia_messstrecke_neu-Zusammenfassung_MultiCurrent.pdf}
    \caption{Messergebnisse der optimierten Messstrecke (Zusammenfassung)}
    \label{dia:messstrecke_neu}
\end{diagram}

\subsection{Messergebnisse: Einfluss der Leitergeometrie}
\label{sec:einfluss_leitergeometrie}

% Kombinierte Diagramme (Dreieck & Parallel) für den direkten Geometrievergleich
\begin{diagram}[H]
    \centering
    \includegraphics[width=1.0\textwidth]{03_Ressourcen/diagramme/dia_2000A_dreieck_parallel/dia_2000A_dreieck_parallel-Zusammenfassung_MultiCurrent.pdf}
    \caption{Vergleich der Leitergeometrien bei 2000\,A}
    \label{dia:2000A_dreieck_parallel}
\end{diagram}

\begin{diagram}[H]
    \centering
    \includegraphics[width=1.0\textwidth]{03_Ressourcen/diagramme/dia_2500A_dreieck_parallel/dia_2500A_dreieck_parallel-Zusammenfassung_MultiCurrent.pdf}
    \caption{Vergleich der Leitergeometrien bei 2500\,A}
    \label{dia:2500A_dreieck_parallel}
\end{diagram}

\begin{diagram}[H]
    \centering
    \includegraphics[width=1.0\textwidth]{03_Ressourcen/diagramme/dia_3000A_dreieck_parallel/dia_3000A_dreieck_parallel-Zusammenfassung_MultiCurrent.pdf}
    \caption{Vergleich der Leitergeometrien bei 3000\,A}
    \label{dia:3000A_dreieck_parallel}
\end{diagram}

\begin{diagram}[H]
    \centering
    \includegraphics[width=1.0\textwidth]{03_Ressourcen/diagramme/dia_4000A_dreieck_parallel/dia_4000A_dreieck_parallel-Zusammenfassung_MultiCurrent.pdf}
    \caption{Vergleich der Leitergeometrien bei 4000\,A}
    \label{dia:4000A_dreieck_parallel}
\end{diagram}

\begin{diagram}[H]
    \centering
    \includegraphics[width=1.0\textwidth]{03_Ressourcen/diagramme/dia_5000A_dreieck_parallel/dia_5000A_dreieck_parallel-Zusammenfassung_MultiCurrent.pdf}
    \caption{Vergleich der Leitergeometrien bei 5000\,A}
    \label{dia:5000A_dreieck_parallel}
\end{diagram}

\subsection{Messergebnisse: Vergleich der Wandlertechnologien}
\label{sec:vergleich_wandlertechnologien}

% Einzelbetrachtungen der Geometrien (fokussiert auf Wandlerverhalten)
\begin{diagram}[H]
    \centering
    \includegraphics[width=1.0\textwidth]{03_Ressourcen/diagramme/dia_2000A_dreieck/dia_2000A_dreieck-Zusammenfassung_MultiCurrent.pdf}
    \caption{Wandlerverhalten im Dreiecksaufbau bei 2000\,A}
    \label{dia:2000A_dreieck}
\end{diagram}

\begin{diagram}[H]
    \centering
    \includegraphics[width=1.0\textwidth]{03_Ressourcen/diagramme/dia_2000A_parallel/dia_2000A_parallel-Zusammenfassung_MultiCurrent.pdf}
    \caption{Wandlerverhalten im Parallelaufbau bei 2000\,A}
    \label{dia:2000A_parallel}
\end{diagram}

\begin{diagram}[H]
    \centering
    \includegraphics[width=1.0\textwidth]{03_Ressourcen/diagramme/dia_2500A_dreieck/dia_2500A_dreieck-Zusammenfassung_MultiCurrent.pdf}
    \caption{Wandlerverhalten im Dreiecksaufbau bei 2500\,A}
    \label{dia:2500A_dreieck}
\end{diagram}

\subsection{Auswertung der Messergebnisse}
\label{sec:auswertung_messergebnisse}