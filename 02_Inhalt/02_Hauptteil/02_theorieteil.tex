% =============================================================================
% KAPITEL 2: THEORIETEIL (02_theorieteil.tex)
% -----------------------------------------------------------------------------
% In diesem Kapitel präsentierst du die theoretischen Grundlagen und den
% aktuellen Stand der Forschung.
% Dies ist die Basis, auf der deine eigene Untersuchung aufbaut.
% =============================================================================

\section{Theoretische Grundlagen und aktueller Forschungsstand}
\label{chap:theorie}

\subsection{Funktionsprinzip induktiver Messstromwandler}
\label{sec:funktionsprinzip_wandler}



[Image of current transformer circuit diagram]


Ein induktiver Messstromwandler dient dazu, hohe Wechselströme aus dem Primärnetz in kleine, messbare Ströme auf der Sekundärseite zu transformieren. Dabei fungiert er als Bindeglied zwischen dem Hochstrombereich und den Mess- oder Schutzeinrichtungen.

Das grundlegende Funktionsprinzip basiert auf der galvanischen Trennung zwischen dem Primär- und dem Sekundärkreis. Dies ermöglicht den Anschluss von standardisierten Messgeräten, Zählern oder Schutzrelais, die üblicherweise für Nennströme von $1\,\text{A}$ oder $5\,\text{A}$ ausgelegt sind, ohne diese dem hohen Potenzial oder den hohen Strömen des Primärleiters auszusetzen.

\subsubsection{Aufbau eines Messstromwandlers}
\label{sec:aufbau_wandler}
Ein Messstromwandler besteht im Wesentlichen aus einem magnetischen Kern, einer Sekundärwicklung und einem Gehäuse zur Isolierung.
Das Gehäuse verfügt häufig über eine Fensteröffnung, durch die der Primärleiter (Schiene oder Kabel) hindurchgeführt wird; dieser bildet somit die Primärwicklung mit einer Windungszahl von $N_1 = 1$. Bei Wandlern für sehr kleine Primärströme ist die Primärwicklung oft fest im Gehäuse integriert und verfügt über separate Anschlussklemmen.

Der Eisenkern besteht aus einem ferromagnetischen Werkstoff mit hoher Permeabilität, um den magnetischen Fluss zu führen. Übliche Materialien sind Silizium-Eisen, Nickel-Eisen oder nanokristalline Legierungen \cite[S. 63]{minkner2020messwandler}. Die Sekundärwicklung ist auf diesen Kern gewickelt und wird an die externen Klemmen geführt.

\subsubsection{Anwendunggebiete}
\label{sec:anwendungsgebiete}
Messstromwandler werden in unterschiedlichen Genauigkeitsklassen gefertigt, die in der Norm DIN EN 61869-2 festgelegt sind \cite[S. 22; Tab. 201–203]{din_en_61869_2_2013}. Diese Klassen definieren die zulässigen Messabweichungen (Strom- und Fehlwinkel) und sind entscheidend für die Auswahl des geeigneten Wandlers.

Für einfache Betriebsmessungen (z. B. Schalttafelanzeigen), bei denen eine grobe Indikation genügt, kommen Klassen mit geringeren Anforderungen (z. B. Klasse 1 oder 3) zum Einsatz. Für Verrechnungszwecke (Energiezähler) sind präzise Wandler der Klassen 0,2S oder 0,5S erforderlich.
Neben der Messaufgabe spielt der Anlagenschutz eine zentrale Rolle. Schutzwandler (z. B. Klasse 5P oder 10P) sind so dimensioniert, dass sie auch bei hohen Überströmen im Kurzschlussfall nicht in die Sättigung gehen, um ein zuverlässiges Auslösen der Schutzgeräte zu gewährleisten.




\subsubsection{Physikalisches Funktionsprinzip}
\label{sec:physikalisches_prinzip}



[Image of transformer vector diagram]


Physikalisch betrachtet arbeitet der Messstromwandler wie ein kurzgeschlossener Transformator. Jeder stromdurchflossene Leiter ist von einem magnetischen Feld umgeben. Nach dem Durchflutungssatz (Ampèresches Gesetz) erzeugt der Strom $I_p$ im Primärleiter eine magnetische Feldstärke $H$.

Da der Primärleiter durch den Eisenkern geführt wird, konzentriert sich das Magnetfeld im Kernmaterial. Es entsteht ein magnetischer Fluss $\Phi$, der durch die Fläche des Kerns und die magnetische Flussdichte $B$ bestimmt wird. Dieser magnetische Wechselfluss durchsetzt die Sekundärwicklung und induziert gemäß dem Induktionsgesetz eine Spannung $u_{ind}$.

Da der Sekundärkreis im Betrieb über eine niederohmige Bürde (Messgerät und Leitungen) nahezu kurzgeschlossen ist, treibt diese induzierte Spannung einen Sekundärstrom $I_s$. Dieser Sekundärstrom erzeugt wiederum einen magnetischen Fluss, der dem des Primärstroms entgegenwirkt (Lenzsche Regel). Im idealen Wandler heben sich die Durchflutungen $\Theta$ nahezu auf:

\begin{equation}
    I_p \cdot N_p \approx I_s \cdot N_s
\end{equation}

Daraus resultiert das Übersetzungsverhältnis, das antiproportional zu den Windungszahlen ist. In der Realität ist jedoch ein kleiner Teil des Primärstroms notwendig, um den magnetischen Fluss im Eisenkern aufrechtzuerhalten (Magnetisierungsstrom), was zu den typischen Übersetzungs- und Winkelfehlern führt.


\textbf{Magnetische Eigenschaften von Materialien}



\subsection{Entstehung und Ausbreitung magnetischer Störfelder in Schaltanlagen}
\label{sec:stoerfelder_schaltanlagen}

Die Fremdfelder die in der Niederspannung auftreten und Messergebnisse beeinflussen können enstehen durch hohe Primärströme ab 2000 A. Dabei ist der Wandler L2 welcher in der Mitte ist von den EInflüssen am stärksten betroffen.



\newpage
\subsection{Physikalische Auswirkungen von Fremdfeldern auf den Wandlerkern}
\label{sec:auswirkung_fremdfelder}

\subsection{Normative Anforderungen an Genauigkeitsklassen}
\label{sec:normative_anforderungen}

\subsection{Messabweichung und Fehlerfortpflanzung}
\label{sec:messabweichung_fehlerfortpflanzung}