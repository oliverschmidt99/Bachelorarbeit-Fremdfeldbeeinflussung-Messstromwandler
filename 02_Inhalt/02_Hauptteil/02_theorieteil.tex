\section{Theoretische Grundlagen und aktueller Forschungsstand}
\label{chap:theorie}

\subsection{Grundlagen Messstromwandler}
\label{sec:grundlagen_wandler}

Ein Stromwandler dient dazu, hohe Wechselströme aus dem Primärnetz in kleine, messbare Ströme auf der Sekundärseite zu transformieren. Dabei fungiert er als Bindeglied zwischen dem Hochstrombereich und den Mess- oder Schutzeinrichtungen. Das grundlegende Funktionsprinzip basiert auf der galvanischen Trennung zwischen dem Primär- und dem Sekundärkreis. Dies ermöglicht den Anschluss von standardisierten Messgeräten, Zählern oder Schutzrelais, die üblicherweise für Nennströme von 1 A oder 5 A ausgelegt sind, ohne diese dem hohen Potenzial oder den hohen Strömen des Primärleiters auszusetzen.

\subsection{Aufbau und Bauformen}
\label{sec:aufbau_wandler}

Konstruktiv setzt sich der betrachtete Messstromwandler im Wesentlichen aus sechs Hauptkomponenten zusammen, die in Abbildung~\ref{pic:aufbau_wandler} dargestellt sind.

\einfuegenBildGross{03_Ressourcen/zeichnungen/aufbau_wandler.drawio.pdf}{Schematischer Aufbau eines Aufsteckstromwandlers}{pic:aufbau_wandler}

Im Niederspannungsbereich fungiert meist eine Kupferschienenanordnung als Primärleiter. Diese weist in der Regel ein Rechteckprofil auf und kann aus mehreren Einzelschienen bestehen. Eine detaillierte Betrachtung der Schienenanordnung erfolgt in Abschnitt~\ref{sec:geometrie_kupferschienen}. Dieser Primärleiter wird durch die Fensteröffnung des Wandlers geführt. Diese Bauform ohne integrierte Primärwicklung bezeichnet man als Durchsteck- oder Aufsteckstromwandler. Sie entspricht physikalisch einer Windungszahl von eins ($N_1 = 1$) und dominiert aufgrund der einfachen Montage im Bereich mittlerer bis hoher Ströme. Um dabei eine zentrierte Leiterführung zu gewährleisten, bieten einige Hersteller spezielle Vorrichtungen an, mit denen sich unterschiedliche Schienengeometrien im Fensterausschnitt fixieren lassen.

Das zentrale Element der Übertragung bildet der Magnetkern. Ist dieser Kern als geschlossener Toroid ohne Luftspalt ausgeführt, spricht man von einem Ringkern. Der Kern besteht aus einem ferromagnetischen Werkstoff mit hoher Permeabilität ($\mu_r$), um die Übertragungsverluste zu minimieren. Die magnetischen Eigenschaften des Kernmaterials bestimmen dabei maßgeblich die Genauigkeit und das Sättigungsverhalten des Wandlers. Als Werkstoffe kommen üblicherweise Siliziumeisen, Nickeleisen oder nanokristalline Legierungen zum Einsatz~\cite[S.~63]{minkner2020messwandler}.

Direkt auf diesen Ringkern ist die Sekundärwicklung aufgebracht. Sie transformiert den magnetischen Fluss zurück in einen elektrischen Strom und ist mit den externen Anschlussklemmen verbunden. Das Gehäuse umschließt schließlich den gesamten Eisenkern samt Sekundärwicklung und gewährleistet die notwendige elektrische Isolation sowie den mechanischen Schutz.

\subsubsection{Unterscheidung zwischen Mess- und Schutzstromwandlern}
\label{sec:unterschiede_mess_schutzwandler}

Stromwandler lassen sich je nach Anwendungszweck in zwei Hauptkategorien unterteilen: Messstromwandler und Schutzstromwandler. Zwar basieren beide auf dem gleichen physikalischen Prinzip, sie unterscheiden sich jedoch maßgeblich durch ihr Sättigungsverhalten. Der Messstromwandler dient primär dazu, Ströme innerhalb des Nennstrombereichs zu erfassen, um beispielsweise Energiezähler zu speisen. Ein entscheidendes Kriterium ist hierbei, dass der Kern bei hohen Überströmen schnell in die Sättigung geht. Dies begrenzt den Sekundärstrom und schützt so die angeschlossene empfindliche Messtechnik vor Zerstörung~\cite[Kap.~2.2]{redur_einmaleins}.

Im Gegensatz dazu wird der Schutzstromwandler verwendet, um Schutzeinrichtungen wie Relais anzusteuern. Er muss gewährleisten, dass der sekundäre Strom auch weit über den Nennbereich hinaus proportional zum Primärstrom bleibt. Der Kern darf also nicht frühzeitig sättigen, damit die Schutzauslösung zuverlässig erfolgt~\cite[Kap.~2.2]{redur_einmaleins}. Eine weitere spezielle Wandlerart stellen die im folgenden Abschnitt behandelten Kompensationswicklungen dar.

\subsubsection{Kompensationswicklungen}
\label{sec:kompensationswicklungen}

Kompensationswicklungen dienen primär dazu, zwei signifikante Störeinflüsse bei Messstromwandlern zu minimieren. Dies sind der Lagefehler, verursacht durch eine exzentrische Positionierung des Primärleiters (siehe Abbildung~\ref{pic:schema_kupferschiene_pos}), sowie die Einwirkung externer Fremdfelder. Die physikalischen Grundlagen zu letzterem Aspekt werden in Kapitel~\ref{sec:stoerfelder_schaltanlagen} detailliert. Eine exzentrische Leiteranordnung tritt in der Praxis häufig auf, insbesondere wenn die Fensteröffnung des Wandlers deutlich größer dimensioniert ist als der Querschnitt der verwendeten Stromschiene. Aufgrund des geometrischen Spielraums ist eine exakte Zentrierung bei der Montage oft nicht gewährleistet.

\einfuegenBildGross{03_Ressourcen/zeichnungen/aufbau_wandler_kompensiert.drawio.pdf}{Schematischer Aufbau eines Wandlers mit zusätzlichen Kompensationswicklungen}{pic:aufbau_wandler_kompensationswicklungen}

Abbildung~\ref{pic:aufbau_wandler_kompensationswicklungen} veranschaulicht das grundlegende Funktionsprinzip. Die Wicklungen $W_1$ bis $W_4$ sind symmetrisch über den Umfang des Eisenkerns verteilt. Technisch wird dies realisiert, indem Teilwicklungen zusätzlich zur eigentlichen Sekundärwicklung $W_0$ aufgebracht und parallel geschaltet werden. Diese Parallelschaltung ermöglicht den Fluss von Ausgleichsströmen zwischen den Segmenten, wodurch lokale Sättigungserscheinungen und Asymmetrien im Magnetfluss kompensiert werden~\cite[S. 77]{minkner2020messwandler}. Es ist zu beachten, dass die Abbildung lediglich die allgemeine Kompensationswicklungstechnik darstellt. In der Praxis setzen Hersteller oft individuelle, teils proprietäre Wicklungsdesigns ein, weshalb die tatsächliche technische Ausführung von dieser schematischen Darstellung abweichen kann.

\einfuegenBildGross{03_Ressourcen/zeichnungen/kupferschiene_positionierung_vergleich.drawio.pdf}{Schematische Darstellung der zentrischen und exzentrischen Positionierung der Kupferschiene}{pic:schema_kupferschiene_pos}

\subsubsection{Normative Anforderungen und Genauigkeitsklassen}
\label{sec:normen_klassen}

Messstromwandler werden in unterschiedlichen Genauigkeitsklassen gefertigt, deren Anforderungen in der DIN EN 61869-2~\cite[S.~22]{din_en_61869_2_2013} festgelegt sind. Diese Klassen definieren die maximal zulässigen Messabweichungen und bilden die Entscheidungsgrundlage für die Auswahl eines Wandlers. Wie bereits erläutert, unterscheiden sich die Anforderungen an Mess- und Schutzwandler grundlegend.

Für Schutzwandler ist das Verhalten bei hohen Kurzschlussströmen entscheidend. Sie werden in die Klassen P (Protection) und PR (Protection, niedrige Remanenz) eingeteilt. Die Bezeichnung der Genauigkeitsklasse setzt sich aus der höchstzulässigen prozentualen Gesamtmessabweichung, dem Kennbuchstaben und dem Genauigkeitsgrenzfaktor (ALF) zusammen~\cite[S.~23--24]{din_en_61869_2_2013}. Die Klasse P beschreibt dabei Standard-Schutzwandler, für die kein Grenzwert für den Remanenzfluss definiert ist. Im Gegensatz dazu kennzeichnet die Klasse PR Schutzwandler mit begrenztem Remanenzfluss. Diese müssen nach dem Abschalten von Fehlerströmen eine geringe Remanenz aufweisen, was in der Praxis häufig durch konstruktive Maßnahmen wie Luftspalte~\cite[S. 82]{minkner2020messwandler} im Kern erreicht wird. Der Genauigkeitsgrenzfaktor gibt an, bis zum Wievielfachen des Nennstroms die Fehlergrenzen eingehalten werden. Eine Bezeichnung wie „5P20“ bedeutet demnach eine Gesamtmessabweichung von 5\,\% bei 20-fachem Nennstrom. Tabelle~\ref{tab:schutzwandler_fehler} zeigt die Grenzwerte für die gängigen Schutzklassen.

\begin{table}[H]
    \centering
    \caption[Grenzwerte für Schutzwandler der Klassen P und PR]{Grenzwerte der Messabweichung für Stromwandler für Schutzzwecke der Klassen P und PR (gemäß DIN EN 61869-2 Tabelle 205~\cite[S.~24]{din_en_61869_2_2013})}
    \label{tab:schutzwandler_fehler}
    \begin{tabular}{lcccc}
        \toprule
        Genauigkeits- & Übersetzung-           & \multicolumn{2}{c}{Fehlwinkel}  & Gesamtmess-                                    \\
        klasse        & messabweichung         & \multicolumn{2}{c}{bei $I_{n}$} & abweichung                                     \\
                      & bei $I_{n}$ ($\pm$ \%) & ($\pm$ min)                     & ($\pm$ centigrad) & bei $ALF \cdot I_{n}$ (\%) \\
        \midrule
        5P und 5PR    & 1                      & 60                              & 1,8               & 5                          \\
        10P und 10PR  & 3                      & --                              & --                & 10                         \\
        \bottomrule
    \end{tabular}
\end{table}

Messwandler untergliedern sich in die Standardklassen 0,1 bis 1, die Sonderklassen 0,2S und 0,5S für präzise Messungen weit unter Nennstrom sowie die Klassen 3 und 5 für weniger genaue Betriebsmessungen. Die Einhaltung der Fehlergrenzen ist dabei an die angeschlossene Bürde gekoppelt. Für die Standard- sowie die Sonderklassen dürfen die Grenzwerte im Bürdenbereich von 25\,\% bis 100\,\% der Bemessungsleistung nicht überschritten werden. Für die Klassen 3 und 5 gilt hingegen ein Bereich von 50\,\% bis 100\,\%~\cite[S.~21]{din_en_61869_2_2013}. Ein wesentliches Kriterium ist die Übersetzungsmessabweichung $\varepsilon$, die die prozentuale Abweichung des Sekundärstroms vom idealen Wert angibt (siehe Tabelle~\ref{tab:stromfehler}).

\begin{table}[H]
    \centering
    \caption[Grenzwerte für die Übersetzungsmessabweichung]{Grenzwerte für die Übersetzungsmessabweichung (gemäß DIN EN 61869-2 Tabelle 201~\cite[S.~22]{din_en_61869_2_2013})}
    \label{tab:stromfehler}
    \begin{tabular}{lcccc}
        \toprule
        Genauigkeits- & \multicolumn{4}{c}{Übersetzungsmessabweichung $\pm$ \%}                    \\
        klasse        & \multicolumn{4}{c}{bei Strom (\% von $I_{n}$)}                             \\
        \cmidrule(lr){2-5}
                      & 5                                                       & 20   & 100 & 120 \\
        \midrule
        0,1           & 0,4                                                     & 0,2  & 0,1 & 0,1 \\
        0,2           & 0,75                                                    & 0,35 & 0,2 & 0,2 \\
        0,5           & 1,5                                                     & 0,75 & 0,5 & 0,5 \\
        1             & 3,0                                                     & 1,5  & 1,0 & 1,0 \\
        \bottomrule
    \end{tabular}
\end{table}

Neben dem Stromfehler ist der Fehlwinkel entscheidend, der die Phasenverschiebung beschreibt. Die Einhaltung der Grenzwerte ist insbesondere für Schutzeinrichtungen mit Richtungsbestimmung von Bedeutung~\cite[S.~51]{din_en_61869_2_2013}. Tabelle~\ref{tab:fehlwinkel} fasst die zulässigen Fehlwinkel für die Klassen 0,1 bis 1 zusammen.

\begin{table}[H]
    \centering
    \caption[Grenzwerte für den Fehlwinkel]{Grenzwerte für den Fehlwinkel (gemäß DIN EN 61869-2 Tabelle 201~\cite[S.~22]{din_en_61869_2_2013})}
    \label{tab:fehlwinkel}
    \begin{tabular}{lcccccccc}
        \toprule
        Genauigkeits- & \multicolumn{8}{c}{Fehlwinkel}                                                                                                          \\
        klasse        & \multicolumn{4}{c}{$\pm$ Minuten}              & \multicolumn{4}{c}{$\pm$ Zentiradiant}                                                 \\
        \cmidrule(lr){2-5} \cmidrule(lr){6-9}
                      & \multicolumn{4}{c}{bei Strom (\% von $I_{n}$)} & \multicolumn{4}{c}{bei Strom (\% von $I_{n}$)}                                         \\
                      & 5                                              & 20                                             & 100 & 120 & 5    & 20   & 100  & 120  \\
        \midrule
        0,1           & 15                                             & 8                                              & 5   & 5   & 0,45 & 0,24 & 0,15 & 0,15 \\
        0,2           & 30                                             & 15                                             & 10  & 10  & 0,90 & 0,45 & 0,30 & 0,30 \\
        0,5           & 90                                             & 45                                             & 30  & 30  & 2,70 & 1,35 & 0,90 & 0,90 \\
        1             & 180                                            & 90                                             & 60  & 60  & 5,40 & 2,70 & 1,80 & 1,80 \\
        \bottomrule
    \end{tabular}
\end{table}

\subsubsection{Ersatzschaltbild eines Messstromwandlers}
\label{sec:ersatzschaltbild}

Ein Messstromwandler entspricht physikalisch einem Transformator, weshalb sich sein Verhalten mithilfe des Transformator-Ersatzschaltbildes beschreiben lässt. In der Darstellung~\ref{esb:wandler_vollständig} sind die Größen der Primärseite auf die Sekundärseite bezogen. Dabei repräsentieren $R'_{p}$ den Widerstand und $L'_{p}$ die Streuinduktivität der Primärseite. Die Sekundärseite wird durch den Wicklungswiderstand $R_{s}$ und die Streuinduktivität $L_{s}$ gebildet. Der Querzweig, bestehend aus dem Eisenverlustwiderstand $R_{FE}$ und der Hauptinduktivität $L_{H}$, modelliert den Eisenkern. Die externe Bürde wird durch $R_{B}$ und $L_{B}$ abgebildet.

\einfuegenBildGross{03_Ressourcen/zeichnungen/esb_wandler_vollständig.drawio.pdf}{Vollständiges Ersatzschaltbild eines Messstromwandlers}{esb:wandler_vollständig}

Bei einem Aufsteckstromwandler bildet der durch die Fensteröffnung geführte Leiter die Primärwicklung. Da dieser Leiteranteil sehr kurz ist und keine Wicklung im herkömmlichen Sinne darstellt, können der ohmsche Widerstand $R'_p$ und die Streuinduktivität $L'_p$ in der Regel vernachlässigt werden. Das daraus resultierende vereinfachte Ersatzschaltbild ist in Abbildung~\ref{esb:wandler_vereinfacht} dargestellt.

\einfuegenBildGross{03_Ressourcen/zeichnungen/esb_wandler_vereinfacht.drawio.pdf}{Vereinfachtes Ersatzschaltbild eines Messstromwandlers}{esb:wandler_vereinfacht}

Theoretisch ließe sich auch die sekundäre Streuinduktivität $L_{s}$ vernachlässigen, vorausgesetzt, der Primärleiter befindet sich zentriert im Eisenkern und die Sekundärwicklung ist gleichmäßig über den Umfang verteilt~\cite[S.~65]{minkner2020messwandler}. Da in der Praxis jedoch weder eine ideale Zentrierung des Leiters noch eine perfekte Wicklungsverteilung durch den Hersteller garantiert werden kann, wird $L_{s}$ in der hier vorliegenden Betrachtung nicht vernachlässigt.

\subsection{Physikalische Grundlagen}
\label{sec:physikalisches_prinzip}

Physikalisch betrachtet arbeitet der Messstromwandler nach dem Prinzip eines kurzgeschlossenen Transformators. Die Wandlung basiert auf der elektromagnetischen Kopplung zwischen dem Primärleiter und der Sekundärwicklung über einen Magnetkern.

\subsubsection{Magnetfelder und Induktion}

Jeder stromdurchflossene Leiter ist von einem magnetischen Kraftfeld umgeben. Nach dem Durchflutungsgesetz (Ampèresches Gesetz) erzeugt der Primärstrom $I_p$ im Leiter eine magnetische Feldstärke $H$. Für einen unendlich langen, geraden Leiter im Abstand $r$ ergibt sich diese zu:

\begin{equation}
    H = \frac{I_p}{2 \cdot \pi \cdot r}
    \label{equ:h_feld_leiter}
\end{equation}

Das Magnetfeld ist ein wesentlicher Bestandteil der elektromagnetischen Wechselwirkung, einer der vier fundamentalen Grundkräfte der Natur. Historisch prägten Michael Faraday durch die Definition von Feldlinien und James Clerk Maxwell durch die mathematische Vereinheitlichung der Elektrodynamik das heutige Verständnis dieser Kraftwirkung \cite{leifi_magnetfeld}.

In einem Messstromwandler konzentriert sich dieser magnetische Fluss $\Phi$ im Weg mit dem geringsten magnetischen Widerstand $R_m$ (Reluktanz), also im hochpermeablen Eisenkern. Dieser Widerstand ist umgekehrt proportional zur relativen Permeabilität $\mu_r$ des Mediums:

\begin{equation}
    R_m = \frac{l}{\mu_0 \cdot \mu_r \cdot A}
    \label{equ:magnetischer_widerstand}
\end{equation}

In einem Medium mit hohem $\mu_r$ führt eine vorhandene magnetische Feldstärke $H$ zu einer wesentlich höheren magnetischen Flussdichte $B$:

\begin{equation}
    B = \mu_0 \cdot \mu_r \cdot H
    \label{equ:magnetische_flussdichte}
\end{equation}

Da Messstromwandler mit Wechselstrom betrieben werden, erzeugt der zeitlich veränderliche magnetische Fluss $\Phi$ im Kern gemäß dem Induktionsgesetz eine Spannung $u_{s}$ in der Sekundärwicklung:

\begin{equation}
    u_{s} = -N_s \cdot \frac{d\Phi}{dt}
    \label{equ:induktion}
\end{equation}

Hierbei entspricht $N_s$ der Windungszahl der Sekundärwicklung. Im Idealfall entspricht die primäre Durchflutung der sekundären Durchflutung ($I_p \cdot N_p = I_s \cdot N_s$). In der Realität wird jedoch ein Teil des Primärstroms als Magnetisierungsstrom $I_\mu$ benötigt, um den Fluss im Kern aufzubauen, was die grundlegende Ursache für Messabweichungen darstellt.

\subsubsection{Magnetische Stoffeigenschaften}

Die magnetischen Eigenschaften von Materie lassen sich anhand ihrer Reaktion auf ein äußeres Magnetfeld charakterisieren. Maßgeblich hierfür ist die relative Permeabilität $\mu_r$, welche das Verhältnis der Flussdichte im Medium zur Flussdichte im Vakuum beschreibt. In Tabelle \ref{tab:magnetische_stoffe} sind die verschiedenen Stoffeigenschaften nach Kuchling zusammengefasst.

\begin{table}[H]
    \centering
    \caption{Magnetische Stoffeigenschaften und ihre Permeabilitätszahlen \cite[Tab. 48]{kuchling2014taschenbuch}}
    \label{tab:magnetische_stoffe}
    \begin{tabular}{@{}lcll@{}}
        \toprule
        \textbf{Eigenschaft} & \textbf{Permeabilität $\mu_r$} & \textbf{Verhalten}        & \textbf{Materialien}   \\ \midrule
        Diamagnetismus       & $\mu_r < 1$                    & Feldschwächung            & Bi, Cu, Ag, Au, $H_2O$ \\
        Paramagnetismus      & $\mu_r > 1$                    & Schwache Verstärkung      & Al, Pt, Mg, Luft       \\
        Ferromagnetismus     & $\mu_r \gg 1$                  & Starke Verstärkung        & Fe, Co, Ni, Mu-Metall  \\
        Ferrimagnetismus     & $\mu_r \gg 1$                  & Permanente Magnetisierung & Ferrite, Magnetit      \\
        Antiferromagnetismus & $\mu_r \approx 1$              & Keine äußere Wirkung      & Mn, Cr, MnO            \\ \bottomrule
    \end{tabular}
\end{table}

Stoffe mit einer Permeabilität von $\mu_r < 1$ werden als diamagnetisch bezeichnet (z. B. Kupfer oder Wasser). In der Elektrotechnik werden Diamagnetika und Paramagnetika meist näherungsweise wie das Vakuum ($\mu_r \approx 1$) behandelt. Für die Funktion von Messstromwandlern sind ferromagnetische Stoffe aufgrund ihrer extrem hohen Permeabilität von entscheidender Bedeutung, da sie den magnetischen Fluss im Kern führen. Da $\mu_r$ bei Ferromagnetika jedoch feldstärkeabhängig ist, führen starke Felder zu Sättigungseffekten, welche die Genauigkeit bei Fremdfeldeinfluss massiv beeinträchtigen können.

\subsubsection{Hysterese}

\subsubsection{Störeinflüsse bei Messstromwandlern}
\label{sec:stoerfelder_schaltanlagen}

Die Genauigkeit von Messstromwandlern im Betrieb wird maßgeblich von externen und systembedingten Faktoren beeinflusst. Die wesentlichen Störeinflüsse lassen sich in drei Kategorien unterteilen: die fehlerhafte Bürdenbeschaltung, die Einwirkung externer magnetischer Fremdfelder sowie die geometrische Positionierung des Primärleiters.

\textbf{Einfluss der Bürde:} 
Die Bürde stellt einen kritischen Betriebsparameter dar. Eine fehlerhafte Bebürdung des Sekundärkreises – etwa durch zu hohe Leitungswiderstände oder zusätzliche Messgeräte – führt dazu, dass der Wandler eine höhere Sekundärspannung induzieren muss, um den Stromfluss aufrechtzuerhalten. Gemäß dem Induktionsgesetz führt dies zu einer höheren magnetischen Flussdichte im Kern. Wird die Bemessungsbürde überschritten, treibt dies den Eisenkern vorzeitig in die Sättigung, was die Messabweichung $\varepsilon$ massiv ansteigen lässt.


\begin{align*}
    U_s(t) &= -N_s \frac{\diff \Phi (t)}{\diff t} \\ 
    \Leftrightarrow \quad \frac{U_s(t)}{-N_s} &= \frac{\diff \Phi(t)}{\diff t} \\
    \Rightarrow \quad \Phi(t) &= -\frac{1}{N_s} \int U_s(t) \, dt \tag{Hallo}
\end{align*}

\begin{equation}
    R_{\text{Kompensation}}(I_s) = \frac{S_{\text{Bürde}}}{I_s^2} - (R_{\text{Kabel}}+R_{\text{Messgerät}})
\end{equation}




\textbf{Magnetische Fremdfelder:} 
In Niederspannungsschaltanlagen liegen die Stromschienen oft eng beieinander. Die hohen Ströme benachbarter Leiter erzeugen magnetische Felder, die das Feld des zu messenden Leiters überlagern. Diese Fremdfelder dringen in den Kern des Wandlers ein und führen zu einer lokalen magnetischen Übersteuerung. Da ferromagnetische Werkstoffe eine nichtlineare Kennlinie aufweisen, resultiert daraus eine asymmetrische Flussverteilung, welche die Proportionalität zwischen Primär- und Sekundärstrom stört.

\textbf{Exzentrische Positionierung:} 
Eine exzentrische Installation der Kupferschiene innerhalb der Fensteröffnung des Wandlers ist in der Praxis aufgrund unterschiedlicher Schienendimensionen häufig unvermeidlich. Obwohl das Ampèresche Gesetz für einen idealen, homogenen Ringkern eine Unabhängigkeit von der Leiterposition postuliert, führen reale Materialinhomogenitäten und Fertigungstoleranzen der Wicklung dazu, dass die Position einen messbaren Einfluss hat. 



Abbildung \ref{pic:sim_wandler_vergleich} zeigt die resultierende asymmetrische Flussverteilung bei exzentrischer Lage oder Fremdfeldeinfluss in einer Simulation. Man erkennt deutlich, dass die Flussdichte in den Bereichen des Kerns, die dem Primärleiter näher liegen oder in denen sich das Fremdfeld konstruktiv überlagert, deutlich höhere Werte annimmt. Dies kann selbst bei moderaten Primärströmen zu lokalen Sättigungseffekten führen, noch bevor der gesamte Kern seine Sättigungsgrenze erreicht.

\einfuegenBildGross{03_Ressourcen/simulation/wandler_mit_kupferschine/wandler_kupferschine_1p.pdf}{Vergleich der Simulationsergebnisse bei zentrischer und exzentrischer Leiteranordnung im Wandler}{pic:sim_wandler_vergleich}

\subsection{Messabweichung und Fehlerfortpflanzung}
\label{sec:messabweichung_fehlerfortpflanzung}

Zur Sicherstellung der normativ geforderten Genauigkeitsklassen wird in der Prüftechnik das Prinzip der Vergleichsmessung angewendet. Hierbei werden der zu prüfende Wandler und ein hochgenaues Referenznormal primärseitig vom identischen Strom durchflossen. Da die Fehlercharakteristik des Normals bekannt und dessen Eigenabweichung vernachlässigbar klein ist, lässt sich die Messabweichung des Prüflings direkt aus der Differenz der sekundärseitigen Ausgangssignale ableiten.



Gemäß DIN EN 61869-2 müssen dabei zwei zentrale Kenngrößen ermittelt werden. Die Übersetzungsmessabweichung $\varepsilon$ beschreibt die prozentuale Differenz der Stromamplituden und wird nach folgender Gleichung berechnet:

\begin{equation}
    \varepsilon = \frac{k_n \cdot I_s - I_p}{I_p} \cdot 100\,\%
\end{equation}

Hierbei entspricht $k_n$ dem Bemessungsübersetzungsverhältnis. Der zweite Parameter ist der Fehlwinkel $\Delta\phi$, der die Phasenverschiebung zwischen dem primären und dem sekundären Stromvektor definiert. Für die Validierung ist zudem die Belastung mit der Nennbürde entscheidend, da die Impedanz des Sekundärkreises die magnetische Aussteuerung des Kerns und somit die Fehlerwerte maßgeblich beeinflusst.

\subsection{Aufbau Niederspannungsschaltanlagen}
\label{sec:aufbau_niederspannungsschaltanlagen}