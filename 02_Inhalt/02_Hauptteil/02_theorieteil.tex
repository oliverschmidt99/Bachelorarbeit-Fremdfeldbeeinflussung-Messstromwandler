\section{Theoretische Grundlagen und aktueller Forschungsstand}
\label{chap:theorie}

\noindent\textcolor{red}{\rule{\textwidth}{1pt}}

\subsection{Grundlagen induktiver Messstromwandler}
\label{sec:grundlagen_wandler}

Ein induktiver Messstromwandler dient dazu, hohe Wechselströme aus dem Primärnetz in kleine, messbare Ströme auf der Sekundärseite zu transformieren. Dabei fungiert er als Bindeglied zwischen dem Hochstrombereich und den Mess- oder Schutzeinrichtungen.
Das grundlegende Funktionsprinzip basiert auf der galvanischen Trennung zwischen dem Primär- und dem Sekundärkreis. Dies ermöglicht den Anschluss von standardisierten Messgeräten, Zählern oder Schutzrelais, die üblicherweise für Nennströme von 1 A oder 5 A ausgelegt sind, ohne diese dem hohen Potenzial oder den hohen Strömen des Primärleiters auszusetzen.

\subsection{Aufbau und Bauformen}
\label{sec:aufbau_wandler}

Konstruktiv setzt sich der betrachtete Messstromwandler im Wesentlichen aus sechs Hauptkomponenten zusammen, die in Abbildung~\ref{pic:aufbau_wandler} dargestellt sind.

\einfuegenBildGross{03_Ressourcen/zeichnungen/aufbau_wandler.drawio.pdf}{Schematischer Aufbau eines Aufsteckstromwandlers}{pic:aufbau_wandler}

Im Niederspannungsbereich fungiert meist eine Kupferschienenanordnung als Primärleiter. Diese weist in der Regel ein Rechteckprofil auf und kann aus mehreren Einzelschienen bestehen; eine detaillierte Betrachtung der Schienenanordnung erfolgt in Abschnitt~\ref{sec:hochstrom_pruefstand} (Hochstromprüfstand). Dieser Primärleiter wird durch die Fensteröffnung des Wandlers geführt. Um dabei eine zentrierte Leiterführung zu gewährleisten, bieten einige Hersteller spezielle Vorrichtungen an, mit denen sich unterschiedliche Schienengeometrien im Fensterausschnitt fixieren lassen.

Das zentrale Element der Übertragung bildet der Magnetkern. Er hat die Aufgabe, den magnetischen Fluss bei geringstmöglichem magnetischen Widerstand zu bündeln und besteht aus einem ferromagnetischen Werkstoff mit hoher Permeabilität ($\mu_r$), um die Übertragungsverluste zu minimieren. Die magnetischen Eigenschaften des Kernmaterials bestimmen dabei maßgeblich die Genauigkeit und das Sättigungsverhalten des Wandlers. Als Werkstoffe kommen üblicherweise Siliziumeisen, Nickeleisen oder nanokristalline Legierungen zum Einsatz~\cite[S.~63]{minkner2020messwandler}.

Direkt auf diesen Kern ist die Sekundärwicklung aufgebracht. Sie transformiert den magnetischen Fluss zurück in einen elektrischen Strom und ist mit den externen Anschlussklemmen verbunden. Das Gehäuse umschließt schließlich den gesamten Eisenkern samt Sekundärwicklung und gewährleistet die notwendige elektrische Isolation sowie den mechanischen Schutz.

\subsubsection{Unterscheidung zwischen Mess- und Schutzstromwandlern}
\label{sec:unterschiede_mess_schutzwandler}

Stromwandler lassen sich je nach Anwendungszweck in zwei Hauptkategorien unterteilen: Messstromwandler und Schutzstromwandler. Zwar basieren beide auf dem gleichen physikalischen Prinzip, sie unterscheiden sich jedoch maßgeblich durch ihr Sättigungsverhalten.\\

Der Messstromwandler dient primär dazu, Ströme innerhalb des Nennstrombereichs zu erfassen, um beispielsweise Energiezähler zu speisen. Ein entscheidendes Kriterium ist hierbei, dass der Kern bei hohen Überströmen (z.\,B. im Kurzschlussfall) schnell in die Sättigung geht. Dies begrenzt den Sekundärstrom und schützt so die angeschlossene empfindliche Messtechnik vor Zerstörung~\cite[Kap.~2.2]{redur_einmaleins}.\\

Im Gegensatz dazu wird der Schutzstromwandler verwendet, um Schutzeinrichtungen (wie Relais) anzusteuern. Er muss gewährleisten, dass der sekundäre Strom auch weit über den Nennbereich hinaus (im Fehlerfall) proportional zum Primärstrom bleibt. Der Kern darf also nicht frühzeitig sättigen, damit die Schutzauslösung zuverlässig erfolgt~\cite[Kap.~2.2]{redur_einmaleins}. Technisch wird dies oft durch Eisenkernmaterialien mit hoher Sättigungsflussdichte oder Luftspalte im Kern realisiert.\\



\subsubsection{Kompensationswicklungen}
\label{sec:kompensationswicklungen}

Kompensationswicklungen werden eingesetzt, um zwei wesentliche Störeinflüsse bei Messstromwandlern zu reduzieren: den Lagefehler durch eine exzentrische Positionierung des Primärleiters (siehe Abbildung~\ref{pic:schema_kupferschiene_pos}) und die Einwirkung von externen Fremdfeldern. Die physikalischen Hintergründe zu letzteren werden in Kapitel~\ref{sec:auswirkung_fremdfelder} erläutert.

Eine exzentrische Anordnung der Stromschiene tritt in der Praxis häufig auf, wenn die Fensteröffnung des Wandlers deutlich größer dimensioniert ist als der Querschnitt der verwendeten Kupferschiene. Durch diesen geometrischen Spielraum ist eine exakte Zentrierung bei der Montage oft nicht gewährleistet.

\einfuegenBildGross{03_Ressourcen/zeichnungen/aufbau_wandler_kompensiert.drawio.pdf}{Schematischer Aufbau eines Wandlers mit zusätzlichen Kompensationswicklungen}{pic:aufbau_wandler_kompensationswicklungen}

Die Abbildung~\ref{pic:aufbau_wandler_kompensationswicklungen} zeigt schematisch, wie diese Wicklungen ($W_1$ bis $W_4$) symmetrisch auf dem Umfang des Eisenkerns verteilt sind. Technisch ist diese Anordnung so realisiert, dass die Teilwicklungen zusätzlich zur eigentlichen Sekundärwicklung aufgebracht und – wie im rechten Teil der Grafik dargestellt – parallel geschaltet werden. Durch diese Parallelschaltung können sich Ausgleichsströme zwischen den Wicklungssegmenten ausbilden, welche lokale Sättigungserscheinungen oder Asymmetrien im Magnetfluss kompensieren.

\einfuegenBildGross{03_Ressourcen/zeichnungen/kupferschiene_positionierung_vergleich.drawio.pdf}{Schematische Darstellung der zentrischen und exzentrischen Positionierung der Kupferschiene}{pic:schema_kupferschiene_pos}

\noindent\textcolor{red}{\rule{\textwidth}{1pt}}
Hier bin ich gerade aktuelle.
\subsubsection{Normative Anforderungen und Genauigkeitsklassen}
\label{sec:normen_klassen}

Messstromwandler werden in unterschiedlichen Genauigkeitsklassen gefertigt, die in der Norm DIN EN 61869-2 festgelegt sind \cite[S. 22; Tab. 201–203]{din_en_61869_2_2013}. Diese Klassen definieren die zulässigen Messabweichungen (Strom- und Fehlwinkel) und sind entscheidend für die Auswahl des geeigneten Wandlers für die jeweilige Anwendung:





%\paragraph{Konstruktive Anpassung an Anlagenanforderungen}
%Die fortschreitende Miniaturisierung von Niederspannungsschaltanlagen führt zu einem Zielkonflikt bei der Auslegung von Messstromwandlern: Während der verfügbare Bauraum stetig abnimmt, bleiben die physikalischen Anforderungen an die \textit{Genauigkeitsklasse} und die treibbare \textit{Bürde} bestehen. Da die Leistungsfähigkeit eines Wandlers direkt mit dem Eisenquerschnitt korreliert, muss das aktive Kernmaterial (Eisen) möglichst effizient im begrenzten Gehäusevolumen untergebracht werden.

%Bei der Verwendung klassischer Ringkerne auf rechteckigen Stromschienen entsteht konstruktionsbedingt ein hoher Anteil an ungenutztem Luftraum zwischen Leiter und Kern, was das Gehäuse unnötig vergrößert. Um diesen Raumverlust zu minimieren und den magnetischen Weg $l_{Fe}$ kurz zu halten, weicht die Bauform moderner Wandlerkerne häufig von der Toroidform ab. Stattdessen kommen ovale oder rechteckige Kerngeometrien zum Einsatz, die der Kontur des Primärleiters (Stromschiene) folgen. Dies ermöglicht es, auch bei engen Platzverhältnissen und großen Schienenquerschnitten die geforderten magnetischen Kennwerte kosteneffizient zu realisieren.

\subsubsection{Ersatzschaltbild eines Messstromwandlers}
\label{sec:ersatzschaltbild}

Ein Messstromwandler entspricht physikalisch einem Transformator, weshalb sich sein Verhalten mithilfe des Transformator-Ersatzschaltbildes beschreiben lässt. In der Darstellung~\ref{esb:wandler_vollständig} sind die Größen der Primärseite auf die Sekundärseite bezogen. Dabei repräsentieren $R'_{p}$ den Widerstand und $L'_{p}$ die Streuinduktivität der Primärseite. Die Sekundärseite wird durch den Wicklungswiderstand $R_{s}$ und die Streuinduktivität $L_{s}$ gebildet. Der Querzweig, bestehend aus dem Eisenverlustwiderstand $R_{FE}$ und der Hauptinduktivität $L_{H}$, modelliert den Eisenkern. Die externe Bürde wird durch $R_{B}$ und $L_{B}$ abgebildet.

\einfuegenBildGross{03_Ressourcen/zeichnungen/esb_wandler_vollständig.drawio.pdf}{Vollständiges Ersatzschaltbild eines Messstromwandlers}{esb:wandler_vollständig}

Bei einem Aufsteckstromwandler bildet der durch die Fensteröffnung geführte Leiter die Primärwicklung. Da dieser Leiteranteil sehr kurz ist und keine Wicklung im herkömmlichen Sinne darstellt, können der ohmsche Widerstand $R'_p$ und die Streuinduktivität $L'_p$ in der Regel vernachlässigt werden. Das daraus resultierende vereinfachte Ersatzschaltbild ist in Abbildung~\ref{esb:wandler_vereinfacht} dargestellt.

\einfuegenBildGross{03_Ressourcen/zeichnungen/esb_wandler_vereinfacht.drawio.pdf}{Vollständiges Ersatzschaltbild eines Messstromwandlers}{esb:wandler_vereinfacht}

Theoretisch ließe sich auch die sekundäre Streuinduktivität $L_{s}$ vernachlässigen, vorausgesetzt, der Primärleiter befindet sich zentriert im Eisenkern und die Sekundärwicklung ist gleichmäßig über den Umfang verteilt~\cite[S.~65]{minkner2020messwandler}. Da in der Praxis jedoch weder eine ideale Zentrierung des Leiters noch eine perfekte Wicklungsverteilung durch den Hersteller garantiert werden kann, wird $L_{s}$ in der hier vorliegenden Betrachtung – wie im vereinfachten Schaltbild ersichtlich – nicht vernachlässigt.


\subsection{Physikalisches Wirkprinzip}
\label{sec:physikalisches_prinzip}

Physikalisch betrachtet arbeitet der Messstromwandler wie ein kurzgeschlossener Transformator. Jeder stromdurchflossene Leiter ist von einem magnetischen Feld umgeben. Nach dem Durchflutungssatz (Ampèresches Gesetz) erzeugt der Strom $I_p$ im Primärleiter eine magnetische Feldstärke $H$.

Da der Primärleiter durch den Eisenkern geführt wird, konzentriert sich das Magnetfeld im Kernmaterial. Es entsteht ein magnetischer Fluss $\Phi$, der durch die Fläche des Kerns und die magnetische Flussdichte $B$ bestimmt wird. Dieser magnetische Wechselfluss durchsetzt die Sekundärwicklung und induziert gemäß dem Induktionsgesetz eine Spannung $u_{ind}$.

Da der Sekundärkreis im Betrieb über eine niederohmige Bürde (Messgerät und Leitungen) nahezu kurzgeschlossen ist, bewirkt diese induzierte Spannung einen Sekundärstrom $I_s$. Dieser Sekundärstrom erzeugt wiederum einen magnetischen Fluss, der dem des Primärstroms entgegenwirkt (Lenzsche Regel). Im idealen Wandler heben sich die Durchflutungen $\Theta$ nahezu auf:

\begin{equation}
    I_p \cdot N_p \approx I_s \cdot N_s
\end{equation}

Daraus resultiert das Übersetzungsverhältnis, das antiproportional zu den Windungszahlen ist. In der Realität ist jedoch ein kleiner Teil des Primärstroms notwendig, um den magnetischen Fluss im Eisenkern aufrechtzuerhalten (Magnetisierungsstrom), was zu den typischen Übersetzungs- und Winkelfehlern führt.

\einfuegenBildGross{03_Ressourcen/simulation/wandler_mit_kupferschine/wandler_kupferschine_1p.pdf}{Vergleich der Simulationsergebnisse bei zentrischer und exzentrischer Leiteranordnung im Wandler}{pic:sim_wandler_vergleich}

\subsection{Entstehung und Ausbreitung magnetischer Störfelder}
\label{sec:stoerfelder_schaltanlagen}

Die Fremdfelder, die in Niederspannungsschaltanlagen auftreten und Messergebnisse beeinflussen können, entstehen primär durch hohe Ströme in benachbarten Leitern (ab ca. 2000 A). In typischen Sammelschienensystemen ist dabei häufig der mittlere Leiter (L2) von den Einflüssen der Außenleiter am stärksten betroffen, da sich hier die Felder der benachbarten Phasen überlagern.

\subsection{Physikalische Auswirkungen von Fremdfeldern auf den Wandlerkern}
\label{sec:auswirkung_fremdfelder}

% Hier kommt dein Text zur Sättigung / Permeabilitätsänderung durch Fremdfelder

\subsection{Messabweichung und Fehlerfortpflanzung}
\label{sec:messabweichung_fehlerfortpflanzung}

% Hier Formeln für den Gesamtfehler