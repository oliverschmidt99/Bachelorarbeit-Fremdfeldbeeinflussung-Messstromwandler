% =============================================================================
% KAPITEL 2: THEORIETEIL (02_theorieteil.tex)
% -----------------------------------------------------------------------------
% In diesem Kapitel präsentierst du die theoretischen Grundlagen und den
% aktuellen Stand der Forschung.
% Dies ist die Basis, auf der deine eigene Untersuchung aufbaut.
% =============================================================================

\section{Theoretische Grundlagen und aktueller Forschungsstand}
\label{chap:theorie}

\subsection{Funktionsprinzip induktiver Messstromwandler}
\label{sec:funktionsprinzip_wandler}

Ein Messstromwandler funktioniert nach dem Induktionsprinzip, so wie ein Transformator.

\subsubsection{Aufbau eines Messstromwandlers}

Ein Messstromwandler besteht aus einem Gehäuse, einem Eisenkern und der Sekundärwicklung. Die Sekundärwicklung wird nach außen geführt und kann an ein Messgerät angeschlossen werden. Das Gehäuse besitzt eine Fensteröffnung, durch die die Primärleiter hindurchgeführt werden.

Der Eisenkern besteht aus einem ferromagnetischen Werkstoff, beispielsweise Silizium-Eisen, Nickel-Eisen oder nanokristallinen Materialien.

Wandler gibt es in verschiedene Varianten, einmal der Standert-Wandler mit einem Eisenkern, der im Nennbereich in Sättigung geht. Dann nach die Kompensiertenwanlder, die Kompensationswicklungen verbaut haben, die die Fremdfelder egalisieren sollen.

\subsubsection{Physikalische Funktionsprinzip}



\subsection{Entstehung und Ausbreitung magnetischer Störfelder in Schaltanlagen}
\label{sec:stoerfelder_schaltanlagen}

\subsection{Physikalische Auswirkungen von Fremdfeldern auf den Wandlerkern}
\label{sec:auswirkung_fremdfelder}

\subsection{Normative Anforderungen an Genauigkeitsklassen}
\label{sec:normative_anforderungen}

\subsection{Messabweichung und Fehlerfortpflanzung}
\label{sec:messabweichung_fehlerfortpflanzung}