\section{Versuchsaufbau und Methodik}
\label{chap:methodik}

Zur Charakterisierung der Messstromwandler wurde ein Hochstrom-Prüfstand eingesetzt, der primäre Wechselströme von bis zu $6000$\,A generieren kann. Dies ermöglichte die Analyse der magnetischen Eigenschaften sowie der Messgenauigkeit unter realitätsnahen Betriebsbedingungen.

Das methodische Vorgehen untergliedert sich in drei Bereiche: Zunächst erläutert Abschnitt~\ref{sec:prinzip_vergleichsmessung} das grundlegende Messprinzip der Fehlerbestimmung. Darauf aufbauend folgt in Abschnitt~\ref{sec:hochstrom_pruefstand} eine detaillierte technische Beschreibung des Prüfstandsaufbaus. Abschließend stellt Abschnitt~\ref{sec:messtechnisches_konzept} das konkrete Versuchskonzept dar, welches die Variation der Prüfparameter (Stromstärke, Bürde, Leiterpositionierung) definiert.

\subsection{Prüfverfahren zur Bestimmung der Genauigkeit}
\label{sec:prinzip_vergleichsmessung}

Die Bestimmung der Genauigkeit eines Messstromwandlers (Prüfling) erfolgt nach dem Prinzip der Vergleichsmessung gegen ein hochgenaues Referenznormal (Normalwandler).

Hierbei werden der Prüfling und der Normalwandler primärseitig vom selben Strom durchflossen. Sekundärseitig werden die Ausgangsströme beider Wandler einer Messbrücke oder einem Vektor-Netzwerkanalysator zugeführt. Da der Fehler des Normalwandlers vernachlässigbar klein und bekannt ist, kann aus der Differenz der beiden Sekundärströme direkt auf die Messabweichung des Prüflings geschlossen werden.

Zu ermitteln sind hierbei zwei zentrale Größen:

\begin{itemize}
    \item Die \textbf{Übersetzungsmessabweichung} $\varepsilon$ (Stromfehler), welche die Amplitudendifferenz beschreibt.
    \item Der \textbf{Fehlwinkel} $\Delta\phi$, welcher die Phasenverschiebung zwischen dem primären und dem sekundären Stromvektor angibt.
\end{itemize}

Um normkonforme Ergebnisse gemäß DIN EN 61869-2 zu gewährleisten, ist es zwingend erforderlich, den Prüfling mit seiner Nennbürde zu belasten, da die Impedanz des Sekundärkreises einen signifikanten Einfluss auf beide Fehlergrößen hat.



\subsection{Hochstrom-Prüfstand}
\label{sec:hochstrom_pruefstand}

Der Hochstrom-Prüfstand dient der Erzeugung und Regelung hoher Wechselströme für thermische und elektrodynamische Untersuchungen an elektrischen Betriebsmitteln. Als Prüflinge fungieren hierbei in der Regel Niederspannungsschaltanlagen oder deren Teilkomponenten, wie beispielsweise Stromschienensysteme und Schaltgeräte, die unter realitätsnahen Lastbedingungen auf ihre thermische Belastbarkeit und Festigkeit geprüft werden. Die Anlage ermöglicht die Bereitstellung von Strömen bis \SI{5000}{A} bei einer geringen Sekundärspannung. Im Folgenden werden der elektrotechnische Aufbau, das darauf abgestimmte Messkonzept sowie die Implementierung der Stromregelung detailliert beschrieben.


\subsubsection{Aufbau und Funktionsweise}
\label{sec:aufbau_funktionsweise}


Der Leistungspfad beginnt primärseitig mit einem motorbetriebenen Säulenstelltransformator der Firma Ruhstrat, der mit einer Leistung von \SI{90}{kVA} als zentrales Stellglied fungiert. Dem Transformator sind Netzdrosseln nachgeschaltete, die zur Entkopplung von Stromspitzen dienen. Die Spannungsstellung erfolgt stufenlos über verstellbare Kohlerollbürsten, wodurch eine variable Ausgangsspannung zwischen \SI{0}{V} und \SI{380}{V} bereitgestellt wird. Diese Spannung speist den nachgeschalteten Hochstrom-Festtransformator von Rolf Janssen (Typ UI 260/420 M). Mit einer Nennleistung von \SI{30}{kVA} transformiert dieser die Spannung auf eine Kleinspannung von \SI{6}{V} herab, was sekundärseitig die Realisierung von Prüfströmen bis zu \SI{5000}{A} am dreiphasigen Abgang ermöglicht.

\einfuegenBildGross{03_Ressourcen/zeichnungen/aufbau_hochstrom_pruefstand.drawio.pdf}{Blockschaltbild des Hochstrom-Prüfstandes mit Leistungs- und Signalpfaden}{pic:aufbau_hochstrom_pruefstand}

Zur Überwachung und Regelung des Systems ist eine Sensorik integriert. Die Erfassung der Ströme erfolgt über Celsa-Stromwandler der Klasse 0,2S, deren Signale durch Messumformer des Typs K-3 (3-K Elektrik) in analoge Normsignale (\SI{0}{\milli\ampere} bis \SI{20}{\milli\ampere}) gewandelt werden. Diese Signale dienen der dezentralen Peripherie Siemens ET 200S als Istwerte für die Stromregelung. Zusätzlich können Rogowskispulen zur Erfassung von Stromverläufen sowie Temperatursensoren zur thermischen Überwachung des Prüflings angeschlossen werden. Die gesamte Steuerung, Parametrierung und das Datenlogging erfolgen über ein HMI-System auf Basis von Siemens WinCC. Eine detaillierte Spezifikation der einzelnen Komponenten ist in Tabelle~\ref{tab:komponenten_hochstrom} zusammengefasst.



\begin{table}[htbp]
    \centering
    \caption{Detaillierte Spezifikation der Komponenten}
    \label{tab:komponenten_hochstrom}
    \small
    \begin{tabular}{@{}l l l l l@{}}
        \toprule
        \textbf{Komponente}  & \textbf{Typ}    & \textbf{Leistung / Bürde}                                             & \textbf{Primär / Input} & \textbf{Sekundär / Output} \\ \midrule

        Säulenstelltrafo     & Ruhstrat        & \SI{90}{kVA}                                                          & \SI{380}{V}             & \SI{0}{V}--\SI{380}{V}     \\
                             &                 &                                                                       &                         & (\SI{70}{A})               \\ \addlinespace

        Festtransformator    & Janssen         & \SI{30}{kVA}                                                          & \SI{380}{V}             & \SI{6}{V}                  \\
                             & UI 260/420 M    &                                                                       &                         & (\SI{5000}{A})             \\ \addlinespace

        Stromwandler         & Celsa ICG       & \SI{5}{VA}                                                            & \SI{6000}{A}            & \SI{5}{A}                  \\
                             &                 &                                                                       &                         & (Kl. 0,2S)                 \\ \addlinespace

        Messumformer (K-3)   & 3-K Elektrik    & --                                                                    & \SI{0}{A}--\SI{5}{A}    & \SI{0}{mA}--\SI{20}{mA}    \\
                             & DMI30-05-V01    &                                                                       & (AC)                    & (DC)                       \\ \midrule

        \textbf{Leittechnik} & \textbf{Typ}    & \multicolumn{3}{l}{\textbf{Beschreibung}}                                                                                    \\ \cmidrule(r){1-2}
        Steuerung            & Siemens ET 200S & \multicolumn{3}{l}{Profinet-Anbindung, Stromregelung, Analogeingänge}                                                        \\
        Visualisierung       & Siemens WinCC   & \multicolumn{3}{l}{HMI-System, Prozessüberwachung, Datenlogging}                                                             \\ \bottomrule
    \end{tabular}
\end{table}


\subsubsection{Messkonzept}
\label{sec:Messkonzept}

Das Messkonzept ist darauf ausgelegt, sowohl die Regelgröße für die SPS bereitzustellen als auch eine umfassende Charakterisierung des Prüflings zu ermöglichen. Wie im Blockschaltbild ersichtlich, erfolgt die Signalerfassung über drei Wege:


\begin{itemize}
    \item \textbf{Stromregelung (Primärkreis):} Drei Celsa-Stromwandler erfassen den Strom im Primärkreis des Festtransformators. Die Wandlersignale (\SI{0}{A} bis \SI{5}{A} AC) werden durch Messumformer des Typs K-3 in ein analoges Normsignal (\SI{0}{mA} bis \SI{20}{mA} DC) umgesetzt. Dieses Signal wird von der Siemens ET 200S eingelesen und dient als Istwert für den Regelalgorithmus.
    \item \textbf{Prüfstromerfassung (Sekundärkreis):} Optional können Rogowski-Spulen am Abgang genutzt werden, um hochfrequente Anteile oder Transienten im Prüfstrom ohne magnetische Sättigungseffekte zu erfassen.
    \item \textbf{Thermische Überwachung:} Über Analogeingänge der ET 200S werden Temperatursensoren (z.\,B. Pt100) direkt am Prüfling oder an kritischen Komponenten des Prüfstandes überwacht.
\end{itemize}

\subsubsection{Regelungskonzept}
\label{sec:regelungskonzept}


Die Stromregelung wird als digitaler PID-Regelkreis innerhalb der Siemens ET 200S realisiert. Der Anwender gibt über die WinCC-Visualisierung den gewünschten Sollstrom vor. Die SPS vergleicht diesen mit dem rückgeführten Istwert der K-3 Messumformer.

Als Stellgröße generiert die Steuerung Schaltbefehle für den Antriebsmotor des Säulenstelltransformators. Um mechanischen Verschleiß durch permanentes Nachregeln (Toter Gang) zu vermeiden, ist eine Hysterese in den Regelalgorithmus integriert. Die Visualisierung übernimmt dabei nicht nur die Parametrierung, sondern auch das lückenlose Datenlogging der Versuchsverläufe.





\subsection{Messtechnisches Konzept und Durchführung}
\label{sec:messtechnisches_konzept}

% 1. Der Teil, in dem du erklärst, wie die elektrischen Werte entstehen
\subsubsection{Versuchsaufbau zur Ermittlung der Messabweichung}
\label{sec:versuchsaufbau_genauigkeit}
% Hier beschreibst du die Referenzmessgeräte, die Stromquelle 
% und das Verfahren (z.B. Vergleichsmethode), um die Werte zu erhalten.

% 2. Der Teil, in dem es um die physische Anordnung geht
\subsubsection{Konstruktive Gestaltung der Primärleiterführung}
\label{sec:geometrie_kupferschienen}

\subsubsubsection{Geometrische Anordnung: Parallel- vs. Dreieckskonfiguration}
% Hier erklärst du die beiden Layouts und warum du sie gewählt hast.

\subsubsubsection{Spezifikation der Kupferschienen und Phasenmittenabstände}
% Hier gehst du auf die Schienenmaße und die Grenzwerte der Abstände ein.