\section{Versuchsaufbau und Methodik}
\label{chap:methodik}

Zur Charakterisierung der Messstromwandler wurde ein Hochstrom-Prüfstand eingesetzt, der primäre Wechselströme von bis zu \SI{6000}{A} generieren kann. Dies ermöglicht die Analyse der magnetischen Eigenschaften sowie der Messgenauigkeit unter realitätsnahen Betriebsbedingungen. Das methodische Vorgehen untergliedert sich in die technische Beschreibung der Basiskomponenten, die kritische Analyse des ursprünglichen Messkonzepts sowie die daraus resultierende Systemoptimierung für die finalen Messreihen.

\subsection{Hochstrom-Prüfstand}
\label{sec:hochstrom_pruefstand}

Der Hochstrom-Prüfstand dient der Erzeugung und Regelung hoher Wechselströme für thermische und elektrodynamische Untersuchungen an elektrischen Betriebsmitteln. Als Prüflinge fungieren in der Regel Niederspannungsschaltanlagen oder deren Teilkomponenten, die unter Lastbedingungen auf ihre thermische Belastbarkeit geprüft werden. Die Anlage ermöglicht die Bereitstellung von Strömen bis \SI{6000}{A} bei einer geringen Sekundärspannung.

\subsubsection{Aufbau und Funktionsweise}
\label{sec:aufbau_funktionsweise}

Der Leistungspfad beginnt primärseitig mit einem motorbetriebenen Säulenstelltransformator der Firma Ruhstrat, der mit einer Leistung von \SI{90}{kVA} als zentrales Stellglied fungiert. Dem Transformator sind Netzdrosseln nachgeschaltet, die zur Entkopplung von Stromspitzen dienen. Die Spannungsstellung erfolgt stufenlos über verstellbare Kohlerollbürsten, wodurch eine variable Ausgangsspannung zwischen \SI{0}{V} und \SI{380}{V} bereitgestellt wird. Diese Spannung speist den nachgeschalteten Hochstrom-Festtransformator von Rolf Janssen (Typ UI 260/420 M). Mit einer Nennleistung von \SI{30}{kVA} transformiert dieser die Spannung auf eine Kleinspannung von \SI{6}{V} herab, was sekundärseitig die Realisierung von Prüfströmen bis zu \SI{6000}{A} am dreiphasigen Abgang ermöglicht.

\einfuegenBildGross{03_Ressourcen/zeichnungen/aufbau_hochstrom_pruefstand.drawio.pdf}{Blockschaltbild des Hochstrom-Prüfstandes mit Leistungs- und Signalpfaden}{pic:aufbau_hochstrom_pruefstand}

\subsubsection{Regelungskonzept}
\label{sec:regelungskonzept}

Die Stromregelung ist als digitaler PID-Regelkreis innerhalb der Siemens ET 200S realisiert. Der Anwender gibt über die WinCC-Visualisierung den gewünschten Sollstrom vor, welchen die SPS kontinuierlich mit dem rückgeführten Istwert vergleicht. Als Stellgröße generiert die Steuerung Schaltbefehle für den Antriebsmotor des Säulenstelltransformators. Um permanentes Nachregeln zu minimieren, die starke Stromspitzen erzeugen, ist eine Hysterese als Totband in den Regelalgorithmus integriert. Die Visualisierung übernimmt dabei neben der Parametrierung auch das lückenlose Datenlogging der Versuchsverläufe. Eine detaillierte Spezifikation der einzelnen Komponenten ist in Tabelle~\ref{tab:komponenten_hochstrom} zusammengefasst.

\begin{table}[H]
    \centering
    \caption{Detaillierte Spezifikation der Komponenten}
    \label{tab:komponenten_hochstrom}
    \small
    \begin{tabular}{@{}l l l l l@{}}
        \toprule
        \textbf{Komponente}  & \textbf{Typ}    & \textbf{Leistung / Bürde} & \textbf{Primär / Input} & \textbf{Sekundär / Output}          \\ \midrule
        Säulenstelltrafo     & Ruhstrat        & \SI{90}{kVA}              & \SI{380}{V}             & \SI{0}{V}--\SI{380}{V} (\SI{70}{A}) \\
        Festtransformator    & Janssen         & \SI{30}{kVA}              & \SI{380}{V}             & \SI{6}{V} (\SI{5000}{A})            \\
        Stromwandler         & Celsa ICG       & \SI{5}{VA}                & \SI{6000}{A}            & \SI{5}{A} (Kl. 0,2S)                \\
        Messumformer (K-3)   & 3-K Elektrik    & --                        & \SI{0}{A}--\SI{5}{A} AC & \SI{0}{mA}--\SI{20}{mA} DC          \\ \midrule
        \textbf{Leittechnik} & \textbf{Typ}    & \multicolumn{3}{l}{\textbf{Beschreibung}}                                              \\ \cmidrule(r){1-2}
        Steuerung            & Siemens ET 200S & \multicolumn{3}{l}{Profinet-Anbindung, Stromregelung, Analogeingänge}                  \\
        Visualisierung       & Siemens WinCC   & \multicolumn{3}{l}{HMI-System, Prozessüberwachung, Datenlogging}                       \\ \bottomrule
    \end{tabular}
\end{table}


\subsubsection{Ursprünglicher Aufbau der Messstrecke}
\label{sec:aufbau_alter_messstrecke}

\einfuegenBildGross{03_Ressourcen/zeichnungen/aufbau_messstrecke.drawio.pdf}{Schematischer Aufbau der Messstrecke zur Ermittlung der Messabweichung}{pic:aufbau_messstrecke}

Der erste Aufbau der Messstrecke (siehe Abbildung \ref{pic:aufbau_messstrecke}) sah zwei parallele Erfassungspfade vor. Zur Bestimmung des Referenzwertes der Einspeisung wurden die K-3-Messumformer genutzt, welche das Signal an die SPS übermittelten. Parallel dazu wurde das Ausgangssignal des Prüflings mit einem Digitalmultimeter von Fluke im Modus „Acquire“ erfasst.
%Die Inbetriebnahme diente dazu, die Funktionalität der Regelung sowie die Genauigkeit der Messwerterfassung unter Lastbedingungen zu verifizieren.

\subsubsection{Schwachstellen im Messkonzept}
\label{sec:fehleranalyse}

Die Auswertung der ersten Messreihen zeigt Abweichungen, die über den gesamten Messbereich auftreten. In Abbildung \ref{pic:dia_messstrecke_alt} sind die Verläufe der Messumformer K-3 (blau) und der Rogowskispulen (orange) dargestellt. Die Rogowskispulen wurden parallel eingesetzt, um deren Eignung für die Prüfung der Wandler zu untersuchen. Beide Systeme erfassten die Messwerte zeitgleich an einem Messpunkt hinter dem Festtransformator. Als Referenz dient die Genauigkeitsklasse 0,2, da die primärseitigen Stromwandler der Klasse 0,2S entsprechen.

Die grafische Darstellung verdeutlicht, dass beide Messsysteme die durch die Norm vorgegebenen Toleranzgrenzen (gestrichelte Linien) nicht einhalten. In Phase L1 überschreitet der Messumformer K-3 ab einer Last von 50\,\% den positiven Grenzwert, während die Rogowskispulen eine negative Abweichung zwischen -1\,\% und -2,5\,\% aufweisen. In Phase L2 zeigt der Messumformer K-3 bei 20\,\% Last einen Abfall der Genauigkeit auf circa -2\,\%, wohingegen die Rogowskispulen eine positive Abweichung von über 1\,\% erreichen. Da die Verläufe der beiden Messsysteme voneinander abweichen, kann kein eindeutiger Referenzwert für den Primärstrom bestimmt werden.

\einfuegenBildGross{03_Ressourcen/diagramme/dia_messstrecke_alt.png}{Analyse der Messabweichung und Standardabweichung der Phasen L1, L2 und L3 im ursprünglichen Versuchsaufbau}{pic:dia_messstrecke_alt}

Die im unteren Teil der Abbildung \ref{pic:dia_messstrecke_alt} dargestellte Standardabweichung zeigt, dass beide Messgeräte im niedrigen Strombereich eine erhöhte Varianz aufweisen. Dies ist darauf zurückzuführen, dass der Säulenstelltransformator bei geringen Strömen einen begrenzten Stellbereich besitzt, woraus eine geringere Stabilität des Stromwertes resultiert aus der Regelung. Mit zunehmender Stromstärke reduziert sich dieses Rauschen im Messsignal. Dabei weisen die K-3-Messumformer eine geringere Schwankungsbreite auf als die Rogowskispulen. Aufgrund dieser Ergebnisse sind beide Messverfahren für eine Referenzmessung im Rahmen der Genauigkeitsprüfung unzureichend. Im folgenden Abschnitt \ref{sec:optimierung} wird die Systemoptimierung erläutert, um eine präzisere Messwerterfassung zu realisieren.



\subsection{Optimiertes Messsystem}
\label{sec:optimierung}

Wie im vorangegangenen Abschnitt dargelegt, erfüllte die ursprüngliche Messstrecke nicht die Anforderungen an eine präzise Genauigkeitsprüfung von Messstromwandlern. Aus diesem Grund wurde eine Optimierung des Hochstrom-Prüfstandes durchgeführt. Das primäre Ziel dieser Maßnahmen war die Steigerung der Messgenauigkeit sowie die Automatisierung des Prüfablaufs, um reproduzierbare und direkt vergleichbare Ergebnisse sicherzustellen.

Zur technischen Umsetzung wurden zwei digitale Energiemessgeräte der Siemens-Produktfamilie ausgewählt. Das Modell PAC3220, welches für die Erfassung des Prüflingsstroms eingesetzt wird \cite[S.~97]{siemens_pac3220_2019}, sowie das zur Überwachung der Einspeisung genutzte PAC4220 \cite[S.~94]{siemens_pac4220_2024} verfügen jeweils über die Genauigkeitsklasse 0,2 für die Strommessung. Diese Geräte ermöglichen durch optionale Erweiterungsmodule eine direkte Integration in die Systemumgebung der Siemens ET 200S via PROFINET-Protokoll. 

Ein wesentlicher Vorteil dieser Konfiguration liegt in der dezentralen Messwerterfassung unmittelbar am Messpunkt. Dadurch lassen sich parasitäre Effekte über lange Messleitungen minimieren. Die Einbindung in die SPS erfolgt durch eine direkte zyklische Datenübertragung, wodurch die manuelle Abfrage von Registern entfällt und eine zeitsynchrone Datenbasis für beide Messpunkte geschaffen wird.



\einfuegenBildGross{03_Ressourcen/zeichnungen/aufbau_hochstrom_pruefstand_new.drawio.pdf}{Schematischer Aufbau der optimierten Messstrecke mit Siemens SENTRON Messgeräten}{pic:aufbau_messstrecke_neu}

Die in Abbildung \ref{pic:aufbau_messstrecke_neu} dargestellte Systemarchitektur verdeutlicht die informationstechnische Vernetzung des optimierten Messaufbaus. Während der obere Teil des Schemas den bereits beschriebenen Leistungspfad darstellt, erfolgt die messtechnische Erfassung der Primärströme zyklisch durch die Geräte PAC 4220 (Einspeisung) und PAC 3220 (Prüfling). 

Die erfassten Stromwerte werden in WinCC archiviert und stehen dort für die Prozessüberwachung zur Verfügung. Über die Benutzeroberfläche (HMI) können diese Daten als CSV-Datei exportiert werden. Diese Datei dient als Grundlage für die anschließende Genauigkeitsmessung und die Auswertung der Versuchsreihen.

\subsubsection{Systemintegration und Programmentwicklung}
\label{sec:integration_programmierung}

Die Einbindung der Messgeräte in die bestehende Umgebung der SPS erfolgt über den Hardware-Konfigurator innerhalb des Simatic Managers. Da die Messgeräte der Reihe SENTRON als Teilnehmer am PROFINET-Netzwerk fungieren, müssen zunächst die entsprechenden GSDML-Dateien in den Hardware-Katalog importiert werden. Anschließend werden die Geräte in der Hardware-Konfiguration vernetzt, wobei jedem Messgerät ein eindeutiger Gerätename sowie eine IP-Adresse zugewiesen wird.

Innerhalb des Hardware-Konfigurators werden die spezifischen Messwert der Geräte auf die Eingangsadressen der SPS abgebildet. Die so bereitgestellten Rohwerte werden im Anwenderprogramm der Steuerung aus dem Peripheriebereich ausgelesen und in DBs übertragen. Dieser Zwischenschritt über die DBs ist notwendig, um die Stromwerte strukturiert für die weitere Programmlogik und die Archivierung bereitzustellen.

Für die Erfassung der Daten in WinCC wird im Variablenhaushalt des Visualisierungssystems eine Kommunikationsverbindung zur Steuerung projektiert. Die Variablen für die gemessenen Ströme werden direkt mit den entsprechenden Speicheradressen innerhalb der DBs verknüpft. Über diese Verbindung werden die Werte der Ströme in regelmäßigen Abständen vom System ausgelesen, auf dem HMI angezeigt und für den späteren Export als CSV-Datei in der Datenbank gespeichert.


Die Automatisierung wird über die SPS realisiert. Hierfür wird der vorhandene PID Regeler verwendet. Für die Genauigkeitsmessung eines Wandlers sind die Messpunkte 5~\%, 20~\%, 100~\% und 120~\% des Primärstroms von bedeutung. Um die Messpunkte anzufahren wird eine Schrittkette entwickelt. Es soll jeden Messpunkte nach einander anfahren und eine Messreihe von 5 min durchführen, damit man ein Mittelwert berechnen kann und das der Kern des Wandler ausreichend belaste wird.

D

\einfuegenBildGross{03_Ressourcen/diagramme/dia_messstrecke_neu.png}{Analyse der Messabweichung und Standardabweichung der Phasen L1, L2 und L3 im neuem Versuchsaufbau}{pic:dia_messstrecke_neu}

\subsubsection{Auswertung der neuen Messstrecke}
\label{sec:auswertung_messstrecke}


\einfuegenBildGross{03_Ressourcen/diagramme/dia_messstrecke_neu.png}{Analyse der Messabweichung und Standardabweichung der Phasen L1, L2 und L3 im neuem Versuchsaufbau}{pic:dia_messstrecke_neu}







\subsection{Konstruktive Gestaltung der Primärleiterführung}
\label{sec:geometrie_kupferschienen}

Zusätzlich zur messtechnischen Optimierung wird die geometrische Anordnung des Primärleiters variiert, um deren Einfluss auf die Fremdfeldbeeinflussung zu untersuchen.

\subsubsection{Geometrische Anordnung: Parallel- vs. Dreieckskonfiguration}
\label{sec:layout_geometrie}
% Hier erfolgt die Erläuterung der Layouts bezüglich der elektromagnetischen Beeinflussung.

\subsubsection{Spezifikation der Kupferschienen und Phasenmittenabstände}
\label{sec:schienen_specs}
% Hier werden Details zu Schienenmaßen und Grenzwerten der Abstände aufgeführt.