\section{Versuchsaufbau und Methodik}
\label{chap:methodik}

Zur Charakterisierung der Messstromwandler wurde ein Hochstrom-Prüfstand eingesetzt, der primäre Wechselströme von bis zu $6000$\,A generieren kann. Dies ermöglichte die Analyse der magnetischen Eigenschaften sowie der Messgenauigkeit unter realitätsnahen Betriebsbedingungen.

Das methodische Vorgehen untergliedert sich in drei Bereiche: Zunächst erläutert Abschnitt~\ref{sec:prinzip_vergleichsmessung} das grundlegende Messprinzip der Fehlerbestimmung. Darauf aufbauend folgt in Abschnitt~\ref{sec:hochstrom_pruefstand} eine detaillierte technische Beschreibung des Prüfstandsaufbaus. Abschließend stellt Abschnitt~\ref{sec:messtechnisches_konzept} das konkrete Versuchskonzept dar, welches die Variation der Prüfparameter (Stromstärke, Bürde, Leiterpositionierung) definiert.

\subsection{Prüfverfahren zur Bestimmung der Genauigkeit}
\label{sec:prinzip_vergleichsmessung}

Die Bestimmung der Genauigkeit eines Messstromwandlers (Prüfling) erfolgt nach dem Prinzip der Vergleichsmessung gegen ein hochgenaues Referenznormal (Normalwandler).

Hierbei werden der Prüfling und der Normalwandler primärseitig vom selben Strom durchflossen. Sekundärseitig werden die Ausgangsströme beider Wandler einer Messbrücke oder einem Vektor-Netzwerkanalysator zugeführt. Da der Fehler des Normalwandlers vernachlässigbar klein und bekannt ist, kann aus der Differenz der beiden Sekundärströme direkt auf die Messabweichung des Prüflings geschlossen werden.

Zu ermitteln sind hierbei zwei zentrale Größen:

\begin{itemize}
    \item Die \textbf{Übersetzungsmessabweichung} $\varepsilon$ (Stromfehler), welche die Amplitudendifferenz beschreibt.
    \item Der \textbf{Fehlwinkel} $\Delta\phi$, welcher die Phasenverschiebung zwischen dem primären und dem sekundären Stromvektor angibt.
\end{itemize}

Um normkonforme Ergebnisse gemäß DIN EN 61869-2 zu gewährleisten, ist es zwingend erforderlich, den Prüfling mit seiner Nennbürde zu belasten, da die Impedanz des Sekundärkreises einen signifikanten Einfluss auf beide Fehlergrößen hat.



\subsection{Hochstrom-Prüfstand}
\label{sec:hochstrom_pruefstand}

\subsubsection{Aufbau und Funktionsweise}

Das Blockschaltbild in Abbildung~\ref{pic:aufbau_hochstrom_pruefstand} veranschaulicht die Topologie des verwendeten Hochstrom-Prüfstandes. Die Anlage wird über eine dreiphasige 160\,A-Einspeisung versorgt und besteht im Wesentlichen aus einer zweistufigen Transformatorenkaskade sowie der zugehörigen Steuerungsperipherie.

Als zentrales Stellglied im Leistungskreis fungiert ein motorbetriebener Säulenstelltransformator der Firma \glqq RPT Ruhstrat Power Technology GmbH\grqq. Dieser stellt die Primärspannung für den nachgeschalteten Hochstromtransformator (Festtransformator) variabel von 0 bis 380\,V bereit. Der Spannungsabgriff erfolgt konstruktiv über zwei auf den Säulen laufende Kohlerollbürsten. An einer Säule sind zwei Außenleiter angeschlossen, wobei die Abgriffe gegenläufig verfahren werden, um die Ausgangsspannung zu variieren.

Die so eingestellte Spannung wird über Netzdrosseln an den Festtransformator weitergeleitet. Dieser transformiert die Spannung auf eine Kleinspannung von 6\,V herab und stellt sekundärseitig die benötigten Prüfströme von bis zu 5000\,A zur Verfügung.

\einfuegenBildGross{03_Ressourcen/zeichnungen/aufbau_hochstrom_pruefstand.drawio.pdf}{Blockschaltbild des Hochstrom-Prüfstandes}{pic:aufbau_hochstrom_pruefstand}

Die technischen Spezifikationen der verwendeten Komponenten sind in Tabelle~\ref{tab:komponenten_hochstrom_v2} zusammengefasst.

\begin{table}[htbp]
    \centering
    \caption{Detaillierte Spezifikation der Komponenten}
    \label{tab:komponenten_hochstrom_v2}
    \small
    \begin{tabular}{@{}p{3cm} p{2.5cm} p{2cm} p{2.5cm} p{2.5cm}@{}}
        \toprule
        \textbf{Komponente}  & \textbf{Typ}                       & \textbf{Leistung / Bürde}                                             & \textbf{Primär / Input}            & \textbf{Sekundär / Output}                   \\ \midrule

        Säulenstelltrafo     & Ruhstrat                           & \SI{90}{kVA}                                                          & \SI{380}{V}                        & \SI{0}{V}--\SI{380}{V} \newline (\SI{70}{A}) \\ \addlinespace

        Festtransformator    & Janssen \newline UI 260/420 M      & \SI{30}{kVA}                                                          & \SI{380}{V}                        & \SI{6}{V} \newline (\SI{5000}{A})            \\ \addlinespace

        Stromwandler         & Celsa ICG                          & \SI{5}{VA}                                                            & \SI{6000}{A}                       & \SI{5}{A} \newline (Kl. 0,2S)                \\ \addlinespace

        Messumformer (K3)    & 3-K Elektrik \newline DMI30-05-V01 & --                                                                    & \SI{0}{A}--\SI{5}{A} \newline (AC) & \SI{0}{mA}--\SI{20}{mA} \newline (DC)        \\ \midrule

        \textbf{Leittechnik} & \textbf{Typ}                       & \multicolumn{3}{l}{\textbf{Beschreibung}}                                                                                                                 \\ \cmidrule(r){1-2}
        Steuerung            & Siemens ET 200S                    & \multicolumn{3}{l}{Profinet-Anbindung, Stromregelung, Analogeingänge}                                                                                     \\
        Visualisierung       & Siemens WinCC                      & \multicolumn{3}{l}{HMI-System, Prozessüberwachung, Logging}                                                                                               \\ \bottomrule
    \end{tabular}
\end{table}


\subsubsection{Messkonzept}
\label{sec:Messkonzept}

Um den Prüfstrom präzise regeln zu können, ist eine exakte Erfassung des Istwertes erforderlich. Die Messkette ist dabei mehrstufig aufgebaut:
Zunächst wird der tatsächliche Hochstrom auf der Sekundärseite des Festtransformators über einen Präzisions-Stromwandler der Klasse 0,2S erfasst. Dieser wandelt den Primärstrom von bis zu \SI{6000}{A} auf ein handhabbares Sekundärsignal von \SI{5}{A} (AC).

Da die analoge Eingangsbaugruppe der Speicherprogrammierbaren Steuerung (SPS) dieses Wechselstromsignal nicht direkt verarbeiten kann, ist eine Signalkonditionierung notwendig. Hierfür wird ein Strommessumformer des Typs K3 (DMI30-05-V01) eingesetzt. Dieser erfüllt zwei Funktionen:

Das resultierende Stromsignal wird schließlich von der Analogeingangskarte der ET 200S eingelesen und digitalisiert.


\subsubsection{Regelungskonzept}
\label{sec:regelungskonzept}

Die Regelung des Prüfstroms erfolgt digital über die Siemens ET 200S Steuerung. In der SPS ist ein Regelalgorithmus implementiert, der als geschlossener Regelkreis arbeitet.

Der digitalisierte Messwert aus dem Messkonzept dient dabei als Istgröße (Rückführgröße). Der Sollwert (Führungsgröße) wird durch den Bediener oder ein automatisiertes Prüfprogramm über das Visualisierungssystem (Siemens WinCC HMI) vorgegeben.
Die SPS berechnet zyklisch die Regelabweichung und steuert daraufhin den Motor des Säulenstelltransformators an. Da der Stelltransformator durch die mechanische Verstellung über den Motor ein integrierendes Verhalten aufweist (der Strom ändert sich so lange, wie der Motor läuft), ist die Regelung entsprechend parametriert, um ein Übersteuern oder Schwingen um den Sollwert zu verhindern.

Zusätzlich übernimmt das WinCC-System das Datenlogging, wodurch der zeitliche Verlauf von Soll- und Iststrom für die Dokumentation der Versuchsergebnisse aufgezeichnet wird.



\subsection{Messtechnisches Konzept und Durchführung}
\label{sec:messtechnisches_konzept}

Für die praktische Durchführung der Messreihen wurde ein systematisches Konzept entwickelt, das sich an den Prüfvorgaben der DIN EN 61869-2 orientiert, diese jedoch um die spezifischen Untersuchungsaspekte dieser Arbeit erweitert.

