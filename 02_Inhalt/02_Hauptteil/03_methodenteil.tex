\section{Versuchsaufbau und Methodik}
\label{chap:methodik}

Zur Charakterisierung der Messstromwandler wurde ein Hochstrom-Prüfstand eingesetzt, der primäre Wechselströme von bis zu $6000$\,A generieren kann. Dies ermöglichte die Analyse der magnetischen Eigenschaften sowie der Messgenauigkeit unter realitätsnahen Betriebsbedingungen.

Das methodische Vorgehen untergliedert sich in drei Bereiche: Zunächst erläutert Abschnitt~\ref{sec:prinzip_vergleichsmessung} das grundlegende Messprinzip der Fehlerbestimmung. Darauf aufbauend folgt in Abschnitt~\ref{sec:hochstrom_pruefstand} eine detaillierte technische Beschreibung des Prüfstandsaufbaus. Abschließend stellt Abschnitt~\ref{sec:messtechnisches_konzept} das konkrete Versuchskonzept dar, welches die Variation der Prüfparameter (Stromstärke, Bürde, Leiterpositionierung) definiert.

\subsection{Prüfverfahren zur Bestimmung der Genauigkeit}
\label{sec:prinzip_vergleichsmessung}

Die Bestimmung der Genauigkeit eines Messstromwandlers (Prüfling) erfolgt nach dem Prinzip der Vergleichsmessung gegen ein hochgenaues Referenznormal (Normalwandler).

Hierbei werden der Prüfling und der Normalwandler primärseitig vom selben Strom durchflossen. Sekundärseitig werden die Ausgangsströme beider Wandler einer Messbrücke oder einem Vektor-Netzwerkanalysator zugeführt. Da der Fehler des Normalwandlers vernachlässigbar klein und bekannt ist, kann aus der Differenz der beiden Sekundärströme direkt auf die Messabweichung des Prüflings geschlossen werden.

Zu ermitteln sind hierbei zwei zentrale Größen:

\begin{itemize}
    \item Die \textbf{Übersetzungsmessabweichung} $\varepsilon$ (Stromfehler), welche die Amplitudendifferenz beschreibt.
    \item Der \textbf{Fehlwinkel} $\Delta\phi$, welcher die Phasenverschiebung zwischen dem primären und dem sekundären Stromvektor angibt.
\end{itemize}

Um normkonforme Ergebnisse gemäß DIN EN 61869-2 zu gewährleisten, ist es zwingend erforderlich, den Prüfling mit seiner Nennbürde zu belasten, da die Impedanz des Sekundärkreises einen signifikanten Einfluss auf beide Fehlergrößen hat.



\subsection{Hochstrom-Prüfstand}
\label{sec:hochstrom_pruefstand}




\subsubsection{Aufbau und Verwendungszweck}

Abbildung~\ref{pic:aufbau_hochstrom_pruefstand} zeigt den Aufbau des Prüfstandes. Die Anlage umfasst eine Netzeinspeisung, einen Säulenstelltransformator der Firma \frqq RPT Ruhstrat Power Technology GmbH\flqq, einen Festtransformator sowie die zugehörige Steuerung. Der Spannungsabgriff am Stelltransformator erfolgt über zwei Kohlerollbürsten, die auf den Säulen laufen. An einer Säule sind zwei Außenleiter angeschlossen, wobei die Abgriffe gegenläufig (entgegengesetzt) verfahren werden. Die abgegriffene Spannung wird über eine Netzdrossel an den Festtransformator weitergeleitet.

\einfuegenBildGross{03_Ressourcen/zeichnungen/aufbau_hochstrom_pruefstand.drawio.pdf}{Aufbau des Hochstrom-Prüfstandes}{pic:aufbau_hochstrom_pruefstand}

\begin{table}[H]
    \centering
    \caption{Komponenten und technische Daten der Hochstromquelle}
    \label{tab:komponenten_hochstrom}
    \begin{tabular}{@{}p{3.5cm}p{4cm}p{6cm}@{}}
        \toprule
        \textbf{Komponente}      & \textbf{Hersteller / Typ}         & \textbf{Technische Daten}                                                                    \\ \midrule
        Säulenstelltransformator & Ruhstrat                          & \SI{90}{kVA}; Prim: \SI{380}{V}; Sek: \SI{0}{V} bis \SI{380}{V}; \SI{70}{A}; Schaltgruppe D0 \\ \addlinespace
        Festtransformator        & Janssen \newline Typ UI 260/420 M & \SI{30}{kVA}; Prim: \SI{380}{V}; Sek: \SI{6}{V}; \SI{5000}{A}                                \\ \addlinespace
        Stromwandler             & Celsa \newline Typ ICG            & Prim: \SI{6000}{A}; Sek: \SI{5}{A}; Bürde: \SI{5}{VA}; Klasse 0,2S                           \\ \addlinespace
        Steuerung                & Siemens ET 200S                   & Profinet-Anbindung, Modularer Aufbau, Stromregelung                                          \\ \addlinespace
        Visualisierung           & Siemens WinCC                     & HMI-System, Prozessüberwachung, Sollwertvorgabe                                              \\ \bottomrule
    \end{tabular}
\end{table}


\subsection{Messtechnisches Konzept und Durchführung}
\label{sec:messtechnisches_konzept}

Für die praktische Durchführung der Messreihen wurde ein systematisches Konzept entwickelt, das sich an den Prüfvorgaben der DIN EN 61869-2 orientiert, diese jedoch um die spezifischen Untersuchungsaspekte dieser Arbeit erweitert.

