% =============================================================================
% HAUPTDOKUMENT (main.tex)
% =============================================================================

% GEÄNDERT: Dokumentenklasse zu 'scrartcl' (Artikel) geändert
% NEU: numbers=noenddot entfernt den Punkt nach der Gliederungsnummer
% NEU: headsepline korrigiert (ohne Wertangabe in der Class-Option)
\documentclass[11pt, a4paper, headsepline, numbers=noenddot]{scrartcl}

% --- EINSTELLUNGEN LADEN ---
% =============================================================================
% PRÄAMBEL (praeambel.tex)
% =============================================================================

% --- GRUNDEINSTELLUNGEN ---
\usepackage[utf8]{inputenc}
\usepackage[T1]{fontenc}
\usepackage[ngerman]{babel}
\usepackage{lmodern}
\usepackage{graphicx}
\usepackage{csquotes}

% --- SEITENLAYOUT ---
% KORRIGIERT: Die fehlerhafte 'headsepline'-Option wurde hier entfernt.
\usepackage[a4paper, left=25mm, right=25mm, top=25mm, bottom=25mm, headheight=3cm, includefoot]{geometry}

% --- KOPF- UND FUSSZEILE (mit scrlayer-scrpage für KOMA-Script) ---
\usepackage{scrlayer-scrpage}
\pagestyle{scrheadings}
\clearpairofpagestyles

% Kopfzeile
\ihead{\includegraphics[width=6.0cm]{03_Ressourcen/Logo/Technik.png}}
\ohead{\Large\bfseries\praktikumstitel}

% Fußzeile
\ifoot{\leftmark}
\ofoot{Seite \thepage\ von \pageref{LastPage}}

% --- LITERATURVERZEICHNIS (BIBLATEX) ---
\usepackage[backend=biber, style=alphabetic, sorting=ynt]{biblatex}

% --- MATHEMATIK ---
\usepackage{amsmath, amssymb, amsthm, latexsym, derivative}

% --- GRAFIKEN & DIAGRAMME ---
\usepackage{epsfig, tikz, subcaption, caption, adjustbox, eso-pic}
\usepackage[european]{circuitikz}
\usepackage{siunitx}
\sisetup{locale = DE}

% --- TABELLEN ---
\usepackage{booktabs, multirow, colortbl}

% --- SONSTIGE PAKETE ---
\usepackage{enumitem}
\usepackage{lastpage}
\usepackage{makeidx}
\makeindex
\usepackage[colorlinks=true, linkcolor=blue, citecolor=green, urlcolor=magenta, hidelinks]{hyperref}
\usepackage{chngcntr}
\usepackage[normalem]{ulem}
\usepackage{calc}
\usepackage{pdfpages}
\usepackage{blindtext}
\usepackage{float}
\usepackage{newfloat}
\usepackage{xcolor}
\usepackage[most]{tcolorbox}
\usepackage[framemethod=TikZ]{mdframed}
\usepackage{etoolbox}

% --- BENUTZERDEFINIERTE EINSTELLUNGEN ---
\setlength{\parindent}{0em}
\setkomafont{sectioning}{\sffamily}

% --- DEINE FLOAT-UMGEBUNGEN ---
\DeclareFloatingEnvironment[listname={Diagrammverzeichnis}, name=Diagramm, fileext=lod, placement=htp]{diagram}
\DeclareFloatingEnvironment[listname={Oszillogrammverzeichnis}, name=Oszillogramm, fileext=loo, placement=htp]{oszillo}

% --- ZÄHLER-EINSTELLUNGEN ---
\counterwithin{figure}{chapter}
\counterwithin{table}{chapter}
\counterwithin{oszillo}{chapter}
\counterwithin{diagram}{chapter}

% --- DEINE BOX-DEFINITIONEN ---
\newmdenv[linecolor=red, linewidth=1pt, frametitle=Hinweis, frametitlebackgroundcolor=red, frametitlefont=\color{white}\bfseries, frametitlerule=true, backgroundcolor=white, roundcorner=2pt]{hinweisbox}
\newmdenv[linecolor=gray, linewidth=1pt, frametitle=Aufgabe, frametitlebackgroundcolor=white, frametitlefont=\color{black}\bfseries, frametitlerule=true, backgroundcolor=white, roundcorner=2pt]{aufgabenbox}
\newcounter{rechnungbox}[section]
\renewcommand{\therechnungbox}{\arabic{rechnungbox}}
\newtcolorbox{rechnungbox}[1][]{colframe=darkgray, colback=white, coltitle=white, colbacktitle=gray, title={Rechnung \therechnungbox}, fonttitle=\bfseries, boxed title style={size=small}, boxrule=1pt, sharp corners=south, before upper=\stepcounter{rechnungbox}, #1}

% --- DEINE TOGGLES ---
\newtoggle{showtoc}
\newtoggle{showtables}
\newtoggle{showfigures}
\newtoggle{showoszillos}
\newtoggle{showdiagrams}
\toggletrue{showtoc}
\togglefalse{showtables}
\toggletrue{showfigures}
\togglefalse{showoszillos}
\togglefalse{showdiagrams}
% =============================================================================
% EIGENE BEFEHLE (befehle.tex)
% -----------------------------------------------------------------------------
% Diese Datei ist für deine persönlichen Makros und Befehle (\newcommand).
% So hältst du die Präambel sauber und kannst deine Befehle zentral verwalten.
% Die globalen Variablen für die Titelseite sind hier ebenfalls definiert.
% =============================================================================

% --- GLOBALE DEFINITIONEN (aus deiner alten main.tex) ---
\newcommand{\praktikumstitel}{Bachelorarbeit}
\newcommand{\semester}{}
\newcommand{\versuchsnummer}{}
\newcommand{\versuchstitel}{Fremdfeldbeeinflussung auf Messstromwandler in der Niederspannung}
\newcommand{\gruppe}{}
\newcommand{\studiengang}{Studiengang Elektrotechnik}
\newcommand{\vornameStudEins}{}
\newcommand{\nachnameStudEins}{}
\newcommand{\matrikelnummerStudEins}{}
\newcommand{\vornameStudZwei}{Oliver}
\newcommand{\nachnameStudZwei}{Schmidt}
\newcommand{\matrikelnummerStudZwei}{7023462}
\newcommand{\betreuerEins}{Dr. Sandro Günter}
\newcommand{\betreuerZwei}{Rainer Ludewig Diplom-Ingenieur (Dipl.Ing.)}

% --- HINTERGRUNDBILD (PFAD ANGEPASST) ---
\newcommand\BackgroundWave{%
    \put(0,0){%
        \parbox[b][\paperheight]{\paperwidth}{%
            \vfill
            \centering
            \vspace{12.0cm}
            \includegraphics[width=\paperwidth]{03_Ressourcen/Logo/hsel-welle-grey}%
            \vfill
        }%
    }%
}
% =============================================================================
% LAYOUT-EINSTELLUNGEN (layout.tex)
% =============================================================================

% --- SEITENRÄNDER UND GRÖSSE DER KOPF-/FUSSZEILE ---
\usepackage[
    a4paper,
    left=25mm,
    right=25mm,
    top=25mm,
    bottom=25mm,
    headheight=2cm,
    footskip=15mm
]{geometry}

% --- KOPF- UND FUSSZEILE (scrlayer-scrpage) ---
\usepackage{scrlayer-scrpage}
\KOMAoptions{headsepline=0.4pt, footsepline=0.4pt} % Korrekte Aktivierung der Linien
\pagestyle{scrheadings}
\clearpairofpagestyles
\automark[section]{section}

% Kopfzeile befüllen
% Vorherige Kopfzeilen-Einstellungen bereinigen
\clearpairofpagestyles

% Links: Logo
\ihead{\includegraphics[height=1.5cm,keepaspectratio]{03_Ressourcen/Logo/logo_rolf-janssen_2024.pdf}}

% Mitte: Überschrift zentriert (eingerahmt von Linien)
\chead{%
    \normalfont\normalsize%
    \vlinie\quad\parbox{8cm}{\centering \versuchstitel}\quad\vlinie%
}

% Rechts: Name
\ohead{%
    \normalfont\normalsize\autorenname%
}

% Fußzeile befüllen
\ifoot{\rightmark}
\ofoot{Seite \thepage\ von \ref{TotPages}} % \ref ist korrekt für totpages

% --- SCHRIFTARTEN ---
\setkomafont{pagehead}{\kopfzeilenSchrift}
\setkomafont{pagefoot}{\fusszeilenSchrift}
\setkomafont{section}{\sectionSchrift}
\setkomafont{subsection}{\subsectionSchrift}

% --- GLOSSAR-EINTRÄGE LADEN ---
\loadglsentries{01_Einstellungen/glossar.tex}

% --- LITERATURQUELLE ANGEBEN ---
\addbibresource{bibliographie.bib}

% -----------------------------------------------------------------------------
% DOKUMENTENBEGINN
% -----------------------------------------------------------------------------

\begin{document}

% ---------- DECKBLATT (ohne Nummer) ----------
\pagenumbering{gobble}
% =============================================================================
% DECKBLATT (00_deckblatt.tex)
% -----------------------------------------------------------------------------
% Dies ist die Titelseite deiner Arbeit.
% =============================================================================

\begin{titlepage}
	% --- KOPFZEILE: LOGO LINKS, LOGO RECHTS + TEXT ---
	\begin{center}
		% Linke Seite: Hochschule Logo
		\begin{minipage}[b]{0.45\textwidth}
			\includegraphics[width=5cm]{03_Ressourcen/Logo/Technik.png}
		\end{minipage}
		\hfill
		% Rechte Seite: Rolf Janssen Logo + Fachbereichstext
		\begin{minipage}[b]{0.45\textwidth}
			% Logo Rolf Janssen (wie gewünscht hier platziert)
			\includegraphics[width=5cm]{03_Ressourcen/Logo/logo_rolf-janssen_2024.png}
		\end{minipage}
	\end{center}

	% --- HAUPTTEIL ---
	\begin{center}
		\vspace{2.5cm}

		% ART DER ARBEIT
		\scshape\LARGE \praktikumstitel

		\vspace{0.5cm}

		% STUDIENGANG
		\normalfont\large im \studiengang

		\vspace{1cm}
		\hrule height 1pt
		\vspace{0.5cm}

		% TITEL DER ARBEIT
		\huge \bfseries \versuchstitel

		\vspace{0.5cm}
		\hrule height 1pt
		\vspace{1cm}

		% AUTOR
		\normalfont \large
		Vorgelegt von:\\
		\vspace{0.2cm}
		\textbf{\vornameStudZwei\ \nachnameStudZwei}\\
		Matrikelnummer: \matrikelnummerStudZwei

		\vspace{1cm}

		% UNTERNEHMEN (REINER TEXT)
		Durchgeführt bei:\\
		\vspace{0.3cm}
		\textbf{Rolf Janssen Elektrotechnische Werke GmbH}\\
		Emsstraße 4\\
		26603 Aurich

	\end{center}

	\vfill

	% --- FUSSZEILE: PRÜFER UND DATUM ---
	\begin{center}
		\normalsize
		\begin{tabular}{rl}
			\textbf{Erstprüfer:}             & \betreuerEins     \\
			\textbf{Zweitprüfer:}            & \betreuerZwei     \\
			                                 &                   \\ % kleiner Abstand
			\textbf{Abgabedatum:}            & Emden, den \today
		\end{tabular}
	\end{center}

\end{titlepage}
\newpage

% ---------- VORSPANN (römische Zählung) ----------
\clearpage
\pagenumbering{Roman}

% --- VERZEICHNISSE ---
% Hier wird nur die Datei geladen. Die Befehle \printglossary stehen 
% IN der Datei '02_verzeichnisse.tex', damit alles an einem Ort ist.
% =============================================================================
% VERZEICHNISSE (02_verzeichnisse.tex)
% -----------------------------------------------------------------------------
% Diese Datei generiert alle Verzeichnisse (Inhalt, Abbildungen etc.)
% basierend auf den Schaltern (\toggletrue/\togglefalse) in der Präambel.
% =============================================================================

% Inhaltsverzeichnis
\iftoggle{showtoc}{
    \hypersetup{linkcolor=black}
    \tableofcontents
    \newpage
}{}

% Tabellenverzeichnis
\iftoggle{showtables}{
    \hypersetup{linkcolor=black}
    \listoftables
    \addcontentsline{toc}{section}{Tabellenverzeichnis}
    \newpage
}{}

% Abbildungsverzeichnis
\iftoggle{showfigures}{
    \listoffigures
    \addcontentsline{toc}{section}{Abbildungsverzeichnis}
    \newpage
}{}

% Oszillogrammverzeichnis
\iftoggle{showoszillos}{
    \listofoszillo
    \addcontentsline{toc}{section}{Oszillogrammverzeichnis}
    \newpage
}{}

% Diagrammverzeichnis
\iftoggle{showdiagrams}{
    \listofdiagram
    \addcontentsline{toc}{section}{Diagrammverzeichnis}
    \newpage
}{}

% --- NEU: ABKÜRZUNGS- UND FORMELZEICHENVERZEICHNIS ---

% Abkürzungsverzeichnis ausgeben
\printglossary[type=\acronymtype, title={Abkürzungsverzeichnis}, toctitle={Abkürzungsverzeichnis}]
\newpage

% Formelzeichenverzeichnis ausgeben
\printglossary[type=main, title={Formelzeichenverzeichnis}, toctitle={Formelzeichenverzeichnis}]
\newpage

% Setzt die Linkfarbe für den Haupttext wieder auf Blau
\hypersetup{linkcolor=blue}
\newpage

% ---------- HAUPTTEIL (arabische Zählung ab 1) ----------
\clearpage
\pagenumbering{arabic}

% =============================================================================
% KAPITEL 1: EINLEITUNG (01_einleitung.tex)
% -----------------------------------------------------------------------------
% Hier führst du den Leser in das Thema ein.
% Du stellst die Relevanz
% dar, formulierst die Forschungsfrage und gibst einen Überblick über
% den Aufbau der Arbeit.
% =============================================================================

\section{Einleitung}
\label{chap:einleitung}

In Niederspannungsschaltanlagen werden immer kompakter gebaut, sodass man in einem Schaltraum möglichst viele Ein und Ausgänge schalten, messen und Anbinden kann, durch die kompakte Bauform werden die Messungen verfälscht durch die enge Baufrom und den hohen Strömen, fangen die Messstromwandler

\subsection{Motivation}
\label{sec:motivation}

\subsection{Problemstellung}
\label{sec:problemstellung}

\subsection{Zielsetzung}
\label{sec:zielsetzung}

\subsection{Vorgehensweise}
\label{sec:vorgehensweise}
\section{Theoretische Grundlagen und aktueller Forschungsstand}
\label{chap:theorie}

\subsection{Grundlagen induktiver Messstromwandler}
\label{sec:grundlagen_wandler}

Ein induktiver Messstromwandler dient dazu, hohe Wechselströme aus dem Primärnetz in kleine, messbare Ströme auf der Sekundärseite zu transformieren. Dabei fungiert er als Bindeglied zwischen dem Hochstrombereich und den Mess- oder Schutzeinrichtungen.
Das grundlegende Funktionsprinzip basiert auf der galvanischen Trennung zwischen dem Primär- und dem Sekundärkreis. Dies ermöglicht den Anschluss von standardisierten Messgeräten, Zählern oder Schutzrelais, die üblicherweise für Nennströme von 1 A oder 5 A ausgelegt sind, ohne diese dem hohen Potenzial oder den hohen Strömen des Primärleiters auszusetzen.

\subsection{Aufbau und Bauformen}
\label{sec:aufbau_wandler}

Konstruktiv setzt sich der betrachtete Messstromwandler im Wesentlichen aus sechs Hauptkomponenten zusammen, die in Abbildung~\ref{pic:aufbau_wandler} dargestellt sind.

\einfuegenBildGross{03_Ressourcen/zeichnungen/aufbau_wandler.drawio.pdf}{Schematischer Aufbau eines Aufsteckstromwandlers}{pic:aufbau_wandler}

Im Niederspannungsbereich fungiert meist eine Kupferschienenanordnung als Primärleiter. Diese weist in der Regel ein Rechteckprofil auf und kann aus mehreren Einzelschienen bestehen; eine detaillierte Betrachtung der Schienenanordnung erfolgt in Abschnitt~\ref{sec:hochstrom_pruefstand} (Hochstromprüfstand). Dieser Primärleiter wird durch die Fensteröffnung des Wandlers geführt. Um dabei eine zentrierte Leiterführung zu gewährleisten, bieten einige Hersteller spezielle Vorrichtungen an, mit denen sich unterschiedliche Schienengeometrien im Fensterausschnitt fixieren lassen.

Das zentrale Element der Übertragung bildet der Magnetkern. Er hat die Aufgabe, den magnetischen Fluss bei geringstmöglichem magnetischen Widerstand zu bündeln und besteht aus einem ferromagnetischen Werkstoff mit hoher Permeabilität ($\mu_r$), um die Übertragungsverluste zu minimieren. Die magnetischen Eigenschaften des Kernmaterials bestimmen dabei maßgeblich die Genauigkeit und das Sättigungsverhalten des Wandlers. Als Werkstoffe kommen üblicherweise Siliziumeisen, Nickeleisen oder nanokristalline Legierungen zum Einsatz~\cite[S.~63]{minkner2020messwandler}.

Direkt auf diesen Kern ist die Sekundärwicklung aufgebracht. Sie transformiert den magnetischen Fluss zurück in einen elektrischen Strom und ist mit den externen Anschlussklemmen verbunden. Das Gehäuse umschließt schließlich den gesamten Eisenkern samt Sekundärwicklung und gewährleistet die notwendige elektrische Isolation sowie den mechanischen Schutz.

%\paragraph{Konstruktive Anpassung an Anlagenanforderungen}
%Die fortschreitende Miniaturisierung von Niederspannungsschaltanlagen führt zu einem Zielkonflikt bei der Auslegung von Messstromwandlern: Während der verfügbare Bauraum stetig abnimmt, bleiben die physikalischen Anforderungen an die \textit{Genauigkeitsklasse} und die treibbare \textit{Bürde} bestehen. Da die Leistungsfähigkeit eines Wandlers direkt mit dem Eisenquerschnitt korreliert, muss das aktive Kernmaterial (Eisen) möglichst effizient im begrenzten Gehäusevolumen untergebracht werden.

%Bei der Verwendung klassischer Ringkerne auf rechteckigen Stromschienen entsteht konstruktionsbedingt ein hoher Anteil an ungenutztem Luftraum zwischen Leiter und Kern, was das Gehäuse unnötig vergrößert. Um diesen Raumverlust zu minimieren und den magnetischen Weg $l_{Fe}$ kurz zu halten, weicht die Bauform moderner Wandlerkerne häufig von der Toroidform ab. Stattdessen kommen ovale oder rechteckige Kerngeometrien zum Einsatz, die der Kontur des Primärleiters (Stromschiene) folgen. Dies ermöglicht es, auch bei engen Platzverhältnissen und großen Schienenquerschnitten die geforderten magnetischen Kennwerte kosteneffizient zu realisieren.

\subsubsection{Ersatzschaltbild eines Messstromwandlers}
\label{sec:ersatzschaltbild}

Ein Messstromwandler entspricht physikalisch einem Transformator, weshalb sich sein Verhalten mithilfe des Transformator-Ersatzschaltbildes beschreiben lässt. In der Darstellung~\ref{esb:wandler_vollständig} sind die Größen der Primärseite auf die Sekundärseite bezogen. Dabei repräsentieren $R'_{p}$ den Widerstand und $L'_{p}$ die Streuinduktivität der Primärseite. Die Sekundärseite wird durch den Wicklungswiderstand $R_{s}$ und die Streuinduktivität $L_{s}$ gebildet. Der Querzweig, bestehend aus dem Eisenverlustwiderstand $R_{FE}$ und der Hauptinduktivität $L_{H}$, modelliert den Eisenkern. Die externe Bürde wird durch $R_{B}$ und $L_{B}$ abgebildet.

\einfuegenBildGross{03_Ressourcen/zeichnungen/esb_wandler_vollständig.drawio.pdf}{Vollständiges Ersatzschaltbild eines Messstromwandlers}{esb:wandler_vollständig}

Bei einem Aufsteckstromwandler bildet der durch die Fensteröffnung geführte Leiter die Primärwicklung. Da dieser Leiteranteil sehr kurz ist und keine Wicklung im herkömmlichen Sinne darstellt, können der ohmsche Widerstand $R'_p$ und die Streuinduktivität $L'_p$ in der Regel vernachlässigt werden. Das daraus resultierende vereinfachte Ersatzschaltbild ist in Abbildung~\ref{esb:wandler_vereinfacht} dargestellt.

Theoretisch ließe sich auch die sekundäre Streuinduktivität $L_{s}$ vernachlässigen, vorausgesetzt, der Primärleiter befindet sich zentriert im Eisenkern und die Sekundärwicklung ist gleichmäßig über den Umfang verteilt~\cite[S.~65]{minkner2020messwandler}. Da in der Praxis jedoch weder eine ideale Zentrierung des Leiters noch eine perfekte Wicklungsverteilung durch den Hersteller garantiert werden kann, wird $L_{s}$ in der hier vorliegenden Betrachtung – wie im vereinfachten Schaltbild ersichtlich – nicht vernachlässigt.

\einfuegenBildGross{03_Ressourcen/zeichnungen/esb_wandler_vereinfacht.drawio.pdf}{Vereinfachtes Ersatzschaltbild eines Messstromwandlers}{esb:wandler_vereinfacht}

\subsection{Physikalisches Wirkprinzip}
\label{sec:physikalisches_prinzip}

Physikalisch betrachtet arbeitet der Messstromwandler wie ein kurzgeschlossener Transformator. Jeder stromdurchflossene Leiter ist von einem magnetischen Feld umgeben. Nach dem Durchflutungssatz (Ampèresches Gesetz) erzeugt der Strom $I_p$ im Primärleiter eine magnetische Feldstärke $H$.

Da der Primärleiter durch den Eisenkern geführt wird, konzentriert sich das Magnetfeld im Kernmaterial. Es entsteht ein magnetischer Fluss $\Phi$, der durch die Fläche des Kerns und die magnetische Flussdichte $B$ bestimmt wird. Dieser magnetische Wechselfluss durchsetzt die Sekundärwicklung und induziert gemäß dem Induktionsgesetz eine Spannung $u_{ind}$.

Da der Sekundärkreis im Betrieb über eine niederohmige Bürde (Messgerät und Leitungen) nahezu kurzgeschlossen ist, bewirkt diese induzierte Spannung einen Sekundärstrom $I_s$. Dieser Sekundärstrom erzeugt wiederum einen magnetischen Fluss, der dem des Primärstroms entgegenwirkt (Lenzsche Regel). Im idealen Wandler heben sich die Durchflutungen $\Theta$ nahezu auf:

\begin{equation}
    I_p \cdot N_p \approx I_s \cdot N_s
\end{equation}

Daraus resultiert das Übersetzungsverhältnis, das antiproportional zu den Windungszahlen ist. In der Realität ist jedoch ein kleiner Teil des Primärstroms notwendig, um den magnetischen Fluss im Eisenkern aufrechtzuerhalten (Magnetisierungsstrom), was zu den typischen Übersetzungs- und Winkelfehlern führt.

\subsection{Normative Anforderungen und Genauigkeitsklassen}
\label{sec:normen_klassen}

Messstromwandler werden in unterschiedlichen Genauigkeitsklassen gefertigt, die in der Norm DIN EN 61869-2 festgelegt sind \cite[S. 22; Tab. 201–203]{din_en_61869_2_2013}. Diese Klassen definieren die zulässigen Messabweichungen (Strom- und Fehlwinkel) und sind entscheidend für die Auswahl des geeigneten Wandlers für die jeweilige Anwendung:

\begin{itemize}
    \item \textbf{Betriebsmessungen:} Für einfache Anzeigen (z. B. Schalttafelanzeigen), bei denen eine grobe Indikation genügt, kommen Klassen mit geringeren Anforderungen (z. B. Klasse 1 oder 3) zum Einsatz.
    \item \textbf{Verrechnungszwecke:} Für Energiezähler sind präzise Wandler der Klassen 0,2S oder 0,5S erforderlich, die auch bei kleinen Teillasten hohe Genauigkeiten aufweisen.
    \item \textbf{Anlagenschutz:} Schutzwandler (z. B. Klasse 5P oder 10P) sind so dimensioniert, dass sie auch bei hohen Überströmen im Kurzschlussfall nicht in die Sättigung gehen, um ein zuverlässiges Auslösen der Schutzgeräte zu gewährleisten.
\end{itemize}

\subsection{Entstehung und Ausbreitung magnetischer Störfelder}
\label{sec:stoerfelder_schaltanlagen}

Die Fremdfelder, die in Niederspannungsschaltanlagen auftreten und Messergebnisse beeinflussen können, entstehen primär durch hohe Ströme in benachbarten Leitern (ab ca. 2000 A). In typischen Sammelschienensystemen ist dabei häufig der mittlere Leiter (L2) von den Einflüssen der Außenleiter am stärksten betroffen, da sich hier die Felder der benachbarten Phasen überlagern.

\subsection{Physikalische Auswirkungen von Fremdfeldern auf den Wandlerkern}
\label{sec:auswirkung_fremdfelder}

% Hier kommt dein Text zur Sättigung / Permeabilitätsänderung durch Fremdfelder

\subsection{Messabweichung und Fehlerfortpflanzung}
\label{sec:messabweichung_fehlerfortpflanzung}

% Hier Formeln für den Gesamtfehler
% =============================================================================
% KAPITEL 3: METHODIK (03_methodenteil.tex)
% -----------------------------------------------------------------------------
% Hier beschreibst du genau, WIE du vorgegangen bist, um deine
% Forschungsfrage zu beantworten (z.B. Versuchsaufbau, Simulation,
% Datenanalyse, Experteninterviews).
% =============================================================================

\section{Versuchsaufbau und Methodik}
\label{chap:methodik}

\subsection{Prinzip der Vergleichsmessung}
\label{sec:prinzip_vergleichsmessung}

\subsection{Beschreibung des Hochstrom-Prüfstands}
\label{sec:beschreibung_hochstrom_pruefstand}

\subsection{Messtechnisches Konzept}
\label{sec:messtechnisches_konzept}

\section{Experimentelle Untersuchung}
\label{chap:methodik}

\subsection{Referenzmessung}

\subsection{Untersuchung 1: Einfluss der Leitergeometrie}

\subsection{Untersuchung 2: Einfluss der Wandlertechnologie}

\subsection{Darstellung der Messabweichungen}
% =============================================================================
% KAPITEL 4: ERGEBNISSE (04_ergebnisteil.tex)
% -----------------------------------------------------------------------------
% In diesem Teil präsentierst du neutral und sachlich die Ergebnisse
% deiner Untersuchung (z.B. Messdaten, Diagramme, Beobachtungen).
% Eine Interpretation findet hier noch nicht statt.
% =============================================================================

\section{Auswertung und Diskussion}
\label{chap:auswertung_diskussion}

\subsection{Durchführung der Messreihen am optimierten Prüfstand}
\label{sec:durchfuehrung_messreihen}

\subsection{Messergebnisse: Einfluss der Leitergeometrie}
\label{sec:einfluss_leitergeometrie}

\subsection{Messergebnisse: Vergleich der Wandlertechnologien}
\label{sec:vergleich_wandlertechnologien}

\subsection{Auswertung der Messergebnisse}
\label{sec:auswertung_messergebnisse}
% =============================================================================
% KAPITEL 5: DISKUSSION & FAZIT (05_diskussion_fazit.tex)
% -----------------------------------------------------------------------------
% Hier interpretierst du die Ergebnisse aus Kapitel 4, setzt sie in
% Bezug zur Theorie aus Kapitel 2, beantwortest deine Forschungsfrage
% und ziehst ein abschließendes Fazit.
% =============================================================================

\section{Zusammenfassung der Ergebnisse}
\label{chap:zusammenfassung_ergebnisse}

\subsection{Bewertung der Prüfstands-Optimierung}
\label{sec:bewertung_pruefstands_optimierung}

\subsection{Konstruktionsempfehlung für zukünftige Wandler-Einbauten}
\label{sec:konstruktionsempfehlung}
\section{Ausblick}
\label{chap:ausblick}

% ===========================================================================
% ANHANG (Nummeriert mit A, B...)
% ===========================================================================
\clearpage
\appendix
\renewcommand{\thesection}{\Alph{section}} % A, B, C ...
\renewcommand{\thesubsection}{\Alph{section}.\arabic{subsection}}

% Anhang einbinden (erzeugt z.B. "A Anhang")
\clearpage
\appendix % Optional: Setzt Nummerierung auf A, B, C...
\section{Messergebnisse}
\label{chap:anhang_content}

In diesem Anhang sind die detaillierten Messwerttabellen der durchgeführten Untersuchungen aufgeführt.

\begin{table}[H]
    \centering
    \setlength{\tabcolsep}{4pt}
    \caption{Messergebnisse: Redur 13A1030.3ffp, 2000\,A, 8,1\,$\Omega$}
    \label{tab:messergebnisse_redur_2000A}
    \small
    \begin{tabular}{
            l
            c
            S[table-format=1.3]
            S[table-format=1.2]
            S[table-format=1.3]
            S[table-format=1.2]
            S[table-format=-1.3]
            S[table-format=1.2]
            S[table-format=-1.3]
            S[table-format=2.2]
        }
        \toprule
        {}          & {}    & \multicolumn{2}{c}{5\,\% $I_n$} & \multicolumn{2}{c}{20\,\% $I_n$} & \multicolumn{2}{c}{100\,\% $I_n$} & \multicolumn{2}{c}{120\,\% $I_n$}                                   \\
        {}          & {}    & \multicolumn{2}{c}{(100\,A)}    & \multicolumn{2}{c}{(400\,A)}     & \multicolumn{2}{c}{(2000\,A)}     & \multicolumn{2}{c}{(2400\,A)}                                       \\
        \cmidrule(lr){3-4} \cmidrule(lr){5-6} \cmidrule(lr){7-8} \cmidrule(lr){9-10}
        {Geometrie} & {Ph.} & {[\%]}                          & {[A]}                            & {[\%]}                            & {[A]}                             & {[\%]} & {[A]} & {[\%]} & {[A]} \\
        \midrule
        Dreieck     & L1    & 0.473                           & 0.47                             & 0.022                             & 0.09                              & -0.120 & 2.40  & -0.121 & 2.90  \\
        Dreieck     & L2    & 0.338                           & 0.34                             & 0.088                             & 0.35                              & -0.094 & 1.88  & -1.533 & 36.79 \\
        Dreieck     & L3    & 0.471                           & 0.47                             & 0.111                             & 0.44                              & 0.126  & 2.52  & -0.070 & 1.68  \\
        \addlinespace
        Parallel    & L1    & 0.376                           & 0.38                             & 0.058                             & 0.23                              & -0.081 & 1.62  & -0.103 & 2.47  \\
        Parallel    & L2    & 0.445                           & 0.45                             & 0.111                             & 0.44                              & -0.029 & 0.58  & -0.509 & 12.22 \\
        Parallel    & L3    & 0.405                           & 0.41                             & 0.146                             & 0.58                              & 0.017  & 0.34  & -0.007 & 0.17  \\
        \bottomrule
    \end{tabular}
\end{table}

\begin{table}[H]
    \centering
    \setlength{\tabcolsep}{4pt}
    \caption{Messergebnisse: Celsa ALO 10030, 2000\,A, 1,35\,$\Omega$}
    \label{tab:messergebnisse_celsa_2000A}
    \small
    \begin{tabular}{
            l
            c
            S[table-format=1.3]
            S[table-format=1.2]
            S[table-format=-1.3]
            S[table-format=1.2]
            S[table-format=-1.3]
            S[table-format=2.2]
            S[table-format=-1.3]
            S[table-format=3.2]
        }
        \toprule
        {}          & {}    & \multicolumn{2}{c}{5\,\% $I_n$} & \multicolumn{2}{c}{20\,\% $I_n$} & \multicolumn{2}{c}{100\,\% $I_n$} & \multicolumn{2}{c}{120\,\% $I_n$}                                    \\
        {}          & {}    & \multicolumn{2}{c}{(100\,A)}    & \multicolumn{2}{c}{(400\,A)}     & \multicolumn{2}{c}{(2000\,A)}     & \multicolumn{2}{c}{(2400\,A)}                                        \\
        \cmidrule(lr){3-4} \cmidrule(lr){5-6} \cmidrule(lr){7-8} \cmidrule(lr){9-10}
        {Geometrie} & {Ph.} & {[\%]}                          & {[A]}                            & {[\%]}                            & {[A]}                             & {[\%]} & {[A]} & {[\%]} & {[A]}  \\
        \midrule
        Dreieck     & L1    & 0.183                           & 0.18                             & -0.103                            & 0.41                              & -0.273 & 5.46  & -0.317 & 7.61   \\
        Dreieck     & L2    & 0.530                           & 0.53                             & -0.026                            & 0.10                              & -0.211 & 4.22  & -0.230 & 5.52   \\
        Dreieck     & L3    & 0.480                           & 0.48                             & 0.072                             & 0.29                              & -0.153 & 3.06  & -0.758 & 18.19  \\
        \addlinespace
        Parallel    & L1    & 0.196                           & 0.20                             & -0.068                            & 0.27                              & -0.477 & 9.54  & -2.671 & 64.10  \\
        Parallel    & L2    & 0.201                           & 0.20                             & -0.014                            & 0.06                              & -1.971 & 39.42 & -4.240 & 101.76 \\
        Parallel    & L3    & 0.401                           & 0.40                             & 0.083                             & 0.33                              & -0.427 & 8.54  & -3.071 & 73.70  \\
        \bottomrule
    \end{tabular}
\end{table}

\begin{table}[H]
    \centering
    \setlength{\tabcolsep}{4pt}
    \caption{Messergebnisse: Celsa ALO 8030 K, 2000\,A, 8,1\,$\Omega$}
    \label{tab:messergebnisse_celsa8030_2000A}
    \small
    \begin{tabular}{
            l
            c
            S[table-format=1.3]
            S[table-format=1.2]
            S[table-format=-1.3]
            S[table-format=1.2]
            S[table-format=-1.3]
            S[table-format=1.2]
            S[table-format=-1.3]
            S[table-format=2.2]
        }
        \toprule
        {}          & {}    & \multicolumn{2}{c}{5\,\% $I_n$} & \multicolumn{2}{c}{20\,\% $I_n$} & \multicolumn{2}{c}{100\,\% $I_n$} & \multicolumn{2}{c}{120\,\% $I_n$}                                   \\
        {}          & {}    & \multicolumn{2}{c}{(100\,A)}    & \multicolumn{2}{c}{(400\,A)}     & \multicolumn{2}{c}{(2000\,A)}     & \multicolumn{2}{c}{(2400\,A)}                                       \\
        \cmidrule(lr){3-4} \cmidrule(lr){5-6} \cmidrule(lr){7-8} \cmidrule(lr){9-10}
        {Geometrie} & {Ph.} & {[\%]}                          & {[A]}                            & {[\%]}                            & {[A]}                             & {[\%]} & {[A]} & {[\%]} & {[A]} \\
        \midrule
        Dreieck     & L1    & 0.114                           & 0.11                             & -0.115                            & 0.46                              & -0.306 & 6.12  & -0.338 & 8.11  \\
        Dreieck     & L2    & 0.220                           & 0.22                             & -0.134                            & 0.54                              & -0.279 & 5.58  & -0.299 & 7.18  \\
        Dreieck     & L3    & 0.232                           & 0.23                             & -0.110                            & 0.44                              & -0.252 & 5.04  & -0.295 & 7.08  \\
        \addlinespace
        Parallel    & L1    & 0.134                           & 0.13                             & -0.222                            & 0.89                              & -0.354 & 7.08  & -0.428 & 10.27 \\
        Parallel    & L2    & 0.091                           & 0.09                             & -0.135                            & 0.54                              & -0.268 & 5.36  & -0.354 & 8.50  \\
        Parallel    & L3    & 0.211                           & 0.21                             & -0.055                            & 0.22                              & -0.200 & 4.00  & -0.217 & 5.21  \\
        \bottomrule
    \end{tabular}
\end{table}

\begin{table}[H]
    \centering
    \setlength{\tabcolsep}{4pt}
    \caption{Messergebnisse: MBS ASK101.4, 2000\,A, 8,1\,$\Omega$}
    \label{tab:messergebnisse_mbs_2000A}
    \small
    \begin{tabular}{
            l
            c
            S[table-format=1.3]
            S[table-format=1.2]
            S[table-format=-1.3]
            S[table-format=1.2]
            S[table-format=-1.3]
            S[table-format=1.2]
            S[table-format=-1.3]
            S[table-format=2.2]
        }
        \toprule
        {}          & {}    & \multicolumn{2}{c}{5\,\% $I_n$} & \multicolumn{2}{c}{20\,\% $I_n$} & \multicolumn{2}{c}{100\,\% $I_n$} & \multicolumn{2}{c}{120\,\% $I_n$}                                   \\
        {}          & {}    & \multicolumn{2}{c}{(100\,A)}    & \multicolumn{2}{c}{(400\,A)}     & \multicolumn{2}{c}{(2000\,A)}     & \multicolumn{2}{c}{(2400\,A)}                                       \\
        \cmidrule(lr){3-4} \cmidrule(lr){5-6} \cmidrule(lr){7-8} \cmidrule(lr){9-10}
        {Geometrie} & {Ph.} & {[\%]}                          & {[A]}                            & {[\%]}                            & {[A]}                             & {[\%]} & {[A]} & {[\%]} & {[A]} \\
        \midrule
        Dreieck     & L1    & 0.133                           & 0.13                             & -0.097                            & 0.39                              & -0.284 & 5.68  & -0.305 & 7.32  \\
        Dreieck     & L2    & -0.279                          & 0.28                             & -0.248                            & 0.99                              & -0.204 & 4.08  & -0.232 & 5.57  \\
        Dreieck     & L3    & 0.016                           & 0.02                             & -0.044                            & 0.18                              & -0.163 & 3.26  & -0.166 & 3.98  \\
        \addlinespace
        Parallel    & L1    & 0.134                           & 0.13                             & -0.222                            & 0.89                              & -0.353 & 7.06  & -0.428 & 10.27 \\
        Parallel    & L2    & 0.091                           & 0.09                             & -0.135                            & 0.54                              & -0.268 & 5.36  & -0.355 & 8.52  \\
        Parallel    & L3    & 0.211                           & 0.21                             & -0.056                            & 0.22                              & -0.200 & 4.00  & -0.217 & 5.21  \\
        \bottomrule
    \end{tabular}
\end{table}

\begin{table}[H]
    \centering
    \caption{Gesamtmessergebnisse: Celsa ALO 10050 K, 2500\,A, 2,8\,$\Omega$ (Alle Lastpunkte)}
    \label{tab:messergebnisse_celsa10050_2500A_alle}
    \setlength{\tabcolsep}{2pt}
    \scriptsize
    \resizebox{\textwidth}{!}{%
        \begin{tabular}{
                l
                c
                S[table-format=-1.3] S[table-format=1.2]
                S[table-format=-1.3] S[table-format=1.2]
                S[table-format=-1.3] S[table-format=1.2]
                S[table-format=-1.3] S[table-format=1.2]
                S[table-format=-1.3] S[table-format=1.2]
                S[table-format=-1.3] S[table-format=1.2]
                S[table-format=-1.3] S[table-format=2.2]
            }
            \toprule
            {}       & {}    & \multicolumn{2}{c}{5\,\%}    & \multicolumn{2}{c}{20\,\%}   & \multicolumn{2}{c}{50\,\%}    & \multicolumn{2}{c}{80\,\%}    & \multicolumn{2}{c}{90\,\%}    & \multicolumn{2}{c}{100\,\%}   & \multicolumn{2}{c}{120\,\%}                                                              \\
            {}       & {}    & \multicolumn{2}{c}{(125\,A)} & \multicolumn{2}{c}{(500\,A)} & \multicolumn{2}{c}{(1250\,A)} & \multicolumn{2}{c}{(2000\,A)} & \multicolumn{2}{c}{(2250\,A)} & \multicolumn{2}{c}{(2500\,A)} & \multicolumn{2}{c}{(3000\,A)}                                                            \\
            \cmidrule(lr){3-4} \cmidrule(lr){5-6} \cmidrule(lr){7-8} \cmidrule(lr){9-10} \cmidrule(lr){11-12} \cmidrule(lr){13-14} \cmidrule(lr){15-16}
            {Geom.}  & {Ph.} & {[\%]}                       & {[A]}                        & {[\%]}                        & {[A]}                         & {[\%]}                        & {[A]}                         & {[\%]}                        & {[A]} & {[\%]} & {[A]} & {[\%]} & {[A]} & {[\%]} & {[A]} \\
            \midrule
            Dreieck  & L1    & -0.029                       & 0.04                         & -0.186                        & 0.93                          & -0.229                        & 2.86                          & -0.255                        & 5.10  & -0.259 & 5.83  & -0.279 & 6.97  & -0.305 & 9.16  \\
            Dreieck  & L2    & 0.009                        & 0.01                         & -0.140                        & 0.70                          & -0.188                        & 2.35                          & -0.222                        & 4.44  & -0.242 & 5.44  & -0.275 & 6.87  & -0.549 & 16.47 \\
            Dreieck  & L3    & 0.102                        & 0.13                         & -0.077                        & 0.38                          & -0.126                        & 1.57                          & -0.171                        & 3.42  & -0.192 & 4.33  & -0.229 & 5.72  & -0.317 & 9.52  \\
            \addlinespace
            Parallel & L1    & 0.081                        & 0.10                         & -0.164                        & 0.82                          & -0.232                        & 2.90                          & -0.273                        & 5.45  & -0.306 & 6.89  & -0.291 & 7.28  & -0.326 & 9.77  \\
            Parallel & L2    & 0.247                        & 0.31                         & -0.104                        & 0.52                          & -0.192                        & 2.40                          & -0.238                        & 4.76  & -0.301 & 6.78  & -0.372 & 9.31  & -1.012 & 30.35 \\
            Parallel & L3    & 0.234                        & 0.29                         & -0.053                        & 0.27                          & -0.135                        & 1.68                          & -0.173                        & 3.45  & -0.199 & 4.49  & -0.197 & 4.93  & -0.264 & 7.92  \\
            \bottomrule
        \end{tabular}
    }
\end{table}

\begin{table}[H]
    \centering
    \caption{Gesamtmessergebnisse: Celsa ALO 10030, 2500\,A, 1,35\,$\Omega$ (Alle Lastpunkte)}
    \label{tab:messergebnisse_celsa10030_2500A_alle}
    \setlength{\tabcolsep}{2pt} % Sehr kleiner Spaltenabstand
    \scriptsize % Sehr kleine Schrift
    \resizebox{\textwidth}{!}{% Skaliert die Tabelle auf Textbreite
        \begin{tabular}{
                l
                c
                S[table-format=1.3] S[table-format=1.2]
                S[table-format=-1.3] S[table-format=1.2]
                S[table-format=-1.3] S[table-format=2.2]
                S[table-format=-2.3] S[table-format=3.2]
                S[table-format=-2.3] S[table-format=3.2]
                S[table-format=-2.3] S[table-format=3.2]
                S[table-format=-2.3] S[table-format=3.2]
            }
            \toprule
            {}       & {}    & \multicolumn{2}{c}{5\,\%}    & \multicolumn{2}{c}{20\,\%}   & \multicolumn{2}{c}{50\,\%}    & \multicolumn{2}{c}{80\,\%}    & \multicolumn{2}{c}{90\,\%}    & \multicolumn{2}{c}{100\,\%}   & \multicolumn{2}{c}{120\,\%}                                                                     \\
            {}       & {}    & \multicolumn{2}{c}{(125\,A)} & \multicolumn{2}{c}{(500\,A)} & \multicolumn{2}{c}{(1250\,A)} & \multicolumn{2}{c}{(2000\,A)} & \multicolumn{2}{c}{(2250\,A)} & \multicolumn{2}{c}{(2500\,A)} & \multicolumn{2}{c}{(3000\,A)}                                                                   \\
            \cmidrule(lr){3-4} \cmidrule(lr){5-6} \cmidrule(lr){7-8} \cmidrule(lr){9-10} \cmidrule(lr){11-12} \cmidrule(lr){13-14} \cmidrule(lr){15-16}
            {Geom.}  & {Ph.} & {[\%]}                       & {[A]}                        & {[\%]}                        & {[A]}                         & {[\%]}                        & {[A]}                         & {[\%]}                        & {[A]}  & {[\%]}  & {[A]}  & {[\%]}  & {[A]}  & {[\%]}  & {[A]}  \\
            \midrule
            Dreieck  & L1    & 0.159                        & 0.20                         & -0.107                        & 0.53                          & -0.247                        & 3.09                          & -1.678                        & 33.56  & -3.337  & 75.08  & -5.066  & 126.65 & -8.278  & 248.33 \\
            Dreieck  & L2    & 0.073                        & 0.09                         & -0.204                        & 1.02                          & -0.316                        & 3.95                          & -1.568                        & 31.35  & -1.763  & 39.67  & -1.834  & 45.85  & -1.950  & 58.49  \\
            Dreieck  & L3    & 0.182                        & 0.23                         & -0.139                        & 0.69                          & -0.256                        & 3.20                          & -1.126                        & 22.53  & -2.232  & 50.22  & -3.452  & 86.29  & -5.437  & 163.12 \\
            \addlinespace
            Parallel & L1    & -0.061                       & 0.08                         & -0.214                        & 1.07                          & -0.653                        & 8.17                          & -7.423                        & 148.46 & -10.056 & 226.26 & -12.370 & 309.25 & -16.238 & 487.15 \\
            Parallel & L2    & 0.060                        & 0.08                         & -0.117                        & 0.59                          & -2.732                        & 34.15                         & -11.571                       & 231.42 & -14.277 & 321.22 & -16.623 & 415.59 & -20.565 & 616.96 \\
            Parallel & L3    & 0.180                        & 0.23                         & 0.031                         & 0.16                          & -0.119                        & 1.48                          & -5.352                        & 107.03 & -7.856  & 176.77 & -10.099 & 252.46 & -13.862 & 415.87 \\
            \bottomrule
        \end{tabular}
    }
\end{table}


\begin{table}[H]
    \centering
    \caption{Gesamtmessergebnisse Celsa ALO 12070 bei 3000\,A und 10,8\,$\Omega$ (Alle Lastpunkte)}
    \label{tab:messergebnisse_celsa12070_3000A_alle}
    \setlength{\tabcolsep}{2pt}
    \scriptsize
    \resizebox{\textwidth}{!}{%
        \begin{tabular}{
                l
                c
                S[table-format=1.3] S[table-format=1.2]
                S[table-format=-1.3] S[table-format=1.2]
                S[table-format=-1.3] S[table-format=1.2]
                S[table-format=-1.3] S[table-format=2.2]
                S[table-format=-1.3] S[table-format=2.2]
                S[table-format=-1.3] S[table-format=2.2]
                S[table-format=-1.3] S[table-format=3.2]
            }
            \toprule
            {}       & {}    & \multicolumn{2}{c}{5\,\%}    & \multicolumn{2}{c}{20\,\%}   & \multicolumn{2}{c}{50\,\%}    & \multicolumn{2}{c}{80\,\%}    & \multicolumn{2}{c}{90\,\%}    & \multicolumn{2}{c}{100\,\%}   & \multicolumn{2}{c}{120\,\%}                                                               \\
            {}       & {}    & \multicolumn{2}{c}{(150\,A)} & \multicolumn{2}{c}{(600\,A)} & \multicolumn{2}{c}{(1500\,A)} & \multicolumn{2}{c}{(2400\,A)} & \multicolumn{2}{c}{(2700\,A)} & \multicolumn{2}{c}{(3000\,A)} & \multicolumn{2}{c}{(3600\,A)}                                                             \\
            \cmidrule(lr){3-4} \cmidrule(lr){5-6} \cmidrule(lr){7-8} \cmidrule(lr){9-10} \cmidrule(lr){11-12} \cmidrule(lr){13-14} \cmidrule(lr){15-16}
            {Geom.}  & {Ph.} & {[\%]}                       & {[A]}                        & {[\%]}                        & {[A]}                         & {[\%]}                        & {[A]}                         & {[\%]}                        & {[A]} & {[\%]} & {[A]} & {[\%]} & {[A]} & {[\%]} & {[A]}  \\
            \midrule
            Dreieck  & L1    & 0.138                        & 0.21                         & -0.174                        & 1.04                          & -0.257                        & 3.86                          & -0.314                        & 7.53  & -0.343 & 9.26  & -0.385 & 11.54 & -0.685 & 24.64  \\
            Dreieck  & L2    & 0.201                        & 0.30                         & -0.093                        & 0.56                          & -0.173                        & 2.60                          & -0.196                        & 4.69  & -0.201 & 5.43  & -0.207 & 6.20  & -0.227 & 8.16   \\
            Dreieck  & L3    & 0.330                        & 0.50                         & 0.059                         & 0.36                          & -0.050                        & 0.75                          & -0.113                        & 2.71  & -0.136 & 3.67  & -0.035 & 1.05  & -0.018 & 0.65   \\
            \addlinespace
            Parallel & L1    & 0.265                        & 0.40                         & -0.195                        & 1.17                          & -0.225                        & 3.38                          & -0.301                        & 7.22  & -0.349 & 9.43  & -0.418 & 12.53 & -1.017 & 36.63  \\
            Parallel & L2    & 0.285                        & 0.43                         & -0.089                        & 0.54                          & -0.158                        & 2.37                          & -1.081                        & 25.94 & -1.890 & 51.02 & -2.924 & 87.73 & -5.426 & 195.33 \\
            Parallel & L3    & 0.253                        & 0.38                         & -0.030                        & 0.18                          & -0.098                        & 1.47                          & -0.138                        & 3.31  & -0.149 & 4.02  & -0.170 & 5.10  & -0.910 & 32.77  \\
            \bottomrule
        \end{tabular}
    }
\end{table}

\begin{table}[H]
    \centering
    \caption{Gesamtmessergebnisse Celsa ALO 12070 K bei 3000\,A und 10,8\,$\Omega$ (Alle Lastpunkte)}
    \label{tab:messergebnisse_celsa12070K_3000A_alle}
    \setlength{\tabcolsep}{2pt}
    \scriptsize
    \resizebox{\textwidth}{!}{%
        \begin{tabular}{
                l
                c
                S[table-format=1.3] S[table-format=1.2]
                S[table-format=-1.3] S[table-format=1.2]
                S[table-format=-1.3] S[table-format=1.2]
                S[table-format=-1.3] S[table-format=1.2]
                S[table-format=-1.3] S[table-format=1.2]
                S[table-format=-1.3] S[table-format=1.2]
                S[table-format=-1.3] S[table-format=2.2]
            }
            \toprule
            {}       & {}    & \multicolumn{2}{c}{5\,\%}    & \multicolumn{2}{c}{20\,\%}   & \multicolumn{2}{c}{50\,\%}    & \multicolumn{2}{c}{80\,\%}    & \multicolumn{2}{c}{90\,\%}    & \multicolumn{2}{c}{100\,\%}   & \multicolumn{2}{c}{120\,\%}                                                              \\
            {}       & {}    & \multicolumn{2}{c}{(150\,A)} & \multicolumn{2}{c}{(600\,A)} & \multicolumn{2}{c}{(1500\,A)} & \multicolumn{2}{c}{(2400\,A)} & \multicolumn{2}{c}{(2700\,A)} & \multicolumn{2}{c}{(3000\,A)} & \multicolumn{2}{c}{(3600\,A)}                                                            \\
            \cmidrule(lr){3-4} \cmidrule(lr){5-6} \cmidrule(lr){7-8} \cmidrule(lr){9-10} \cmidrule(lr){11-12} \cmidrule(lr){13-14} \cmidrule(lr){15-16}
            {Geom.}  & {Ph.} & {[\%]}                       & {[A]}                        & {[\%]}                        & {[A]}                         & {[\%]}                        & {[A]}                         & {[\%]}                        & {[A]} & {[\%]} & {[A]} & {[\%]} & {[A]} & {[\%]} & {[A]} \\
            \midrule
            Dreieck  & L1    & 0.273                        & 0.41                         & -0.093                        & 0.56                          & -0.170                        & 2.54                          & -0.184                        & 4.43  & -0.163 & 4.41  & -0.141 & 4.23  & -0.076 & 2.75  \\
            Dreieck  & L2    & 0.296                        & 0.44                         & 0.022                         & 0.13                          & -0.073                        & 1.10                          & -0.108                        & 2.58  & -0.088 & 2.38  & -0.073 & 2.20  & 0.097  & 3.48  \\
            Dreieck  & L3    & 0.338                        & 0.51                         & 0.153                         & 0.92                          & 0.137                         & 2.05                          & 0.169                         & 4.06  & -0.037 & 1.01  & -0.045 & 1.35  & -0.046 & 1.65  \\
            \addlinespace
            Parallel & L1    & 0.194                        & 0.29                         & -0.070                        & 0.42                          & -0.176                        & 2.64                          & -0.174                        & 4.18  & -0.150 & 4.05  & -0.118 & 3.55  & -0.072 & 2.59  \\
            Parallel & L2    & 0.209                        & 0.31                         & -0.009                        & 0.05                          & -0.099                        & 1.48                          & -0.071                        & 1.70  & 0.022  & 0.58  & 0.266  & 7.98  & 0.929  & 33.44 \\
            Parallel & L3    & 0.345                        & 0.52                         & 0.110                         & 0.66                          & 0.001                         & 0.01                          & -0.031                        & 0.73  & -0.018 & 0.50  & 0.012  & 0.37  & 0.082  & 2.95  \\
            \bottomrule
        \end{tabular}
    }
\end{table}

\begin{table}[H]
    \centering
    \caption{Messergebnisse: Celsa ALO 12070, 3000\,A, Asymmetrische Bürde ($R_{L2} = 0\,\Omega$, Parallel)}
    \label{tab:messergebnisse_celsa12070_3000A_asym}
    \setlength{\tabcolsep}{2pt}
    \scriptsize
    \resizebox{\textwidth}{!}{%
        \begin{tabular}{
                l
                c
                S[table-format=1.3] S[table-format=1.2]
                S[table-format=-1.3] S[table-format=1.2]
                S[table-format=-1.3] S[table-format=1.2]
                S[table-format=-1.3] S[table-format=2.2]
                S[table-format=-1.3] S[table-format=2.2]
                S[table-format=-1.3] S[table-format=2.2]
                S[table-format=-1.3] S[table-format=3.2]
            }
            \toprule
            {}       & {}    & \multicolumn{2}{c}{5\,\%}    & \multicolumn{2}{c}{20\,\%}   & \multicolumn{2}{c}{50\,\%}    & \multicolumn{2}{c}{80\,\%}    & \multicolumn{2}{c}{90\,\%}    & \multicolumn{2}{c}{100\,\%}   & \multicolumn{2}{c}{120\,\%}                                                                     \\
            {}       & {}    & \multicolumn{2}{c}{(150\,A)} & \multicolumn{2}{c}{(600\,A)} & \multicolumn{2}{c}{(1500\,A)} & \multicolumn{2}{c}{(2400\,A)} & \multicolumn{2}{c}{(2700\,A)} & \multicolumn{2}{c}{(3000\,A)} & \multicolumn{2}{c}{(3600\,A)}                                                                   \\
            \cmidrule(lr){3-4} \cmidrule(lr){5-6} \cmidrule(lr){7-8} \cmidrule(lr){9-10} \cmidrule(lr){11-12} \cmidrule(lr){13-14} \cmidrule(lr){15-16}
            {Geom.}  & {Ph.} & {[\%]}                       & {[A]}                        & {[\%]}                        & {[A]}                         & {[\%]}                        & {[A]}                         & {[\%]}                        & {[A]}  & {[\%]}  & {[A]}  & {[\%]}  & {[A]}  & {[\%]}  & {[A]}   \\
            \midrule
            Parallel & L1    & 0.063                        & 0.09                         & -0.157                        & 0.94                          & -0.225                        & 3.38                          & -0.306                        & 7.34   & -0.348  & 9.39   & -0.417  & 12.51  & -1.043  & 37.56   \\
            Parallel & L2    & 0.191                        & 0.29                         & -0.061                        & 0.37                          & -0.127                        & 1.91                          & -0.817                        & 19.61  & -1.805  & 48.73  & -3.044  & 91.33  & -5.683  & 204.61  \\
            Parallel & L3    & 0.271                        & 0.41                         & -0.020                        & 0.12                          & -0.107                        & 1.61                          & -0.139                        & 3.34   & -0.151  & 4.07   & -0.163  & 4.90   & -0.726  & 26.13   \\
            \bottomrule
        \end{tabular}
    }
\end{table}

\begin{table}[H]
    \centering
    \caption{Messergebnisse: Celsa ALO 12070, 3000\,A, Minimale Bürde ($R_B \approx 0\,\Omega$, Parallel)}
    \label{tab:messergebnisse_celsa12070_3000A_min}
    \setlength{\tabcolsep}{2pt}
    \scriptsize
    \resizebox{\textwidth}{!}{%
        \begin{tabular}{
                l
                c
                S[table-format=1.3] S[table-format=1.2]
                S[table-format=-1.3] S[table-format=1.2]
                S[table-format=-1.3] S[table-format=1.2]
                S[table-format=-1.3] S[table-format=2.2]
                S[table-format=-1.3] S[table-format=2.2]
                S[table-format=-1.3] S[table-format=2.2]
                S[table-format=-1.3] S[table-format=3.2]
            }
            \toprule
            {}       & {}    & \multicolumn{2}{c}{5\,\%}    & \multicolumn{2}{c}{20\,\%}   & \multicolumn{2}{c}{50\,\%}    & \multicolumn{2}{c}{80\,\%}    & \multicolumn{2}{c}{90\,\%}    & \multicolumn{2}{c}{100\,\%}   & \multicolumn{2}{c}{120\,\%}                                                                     \\
            {}       & {}    & \multicolumn{2}{c}{(150\,A)} & \multicolumn{2}{c}{(600\,A)} & \multicolumn{2}{c}{(1500\,A)} & \multicolumn{2}{c}{(2400\,A)} & \multicolumn{2}{c}{(2700\,A)} & \multicolumn{2}{c}{(3000\,A)} & \multicolumn{2}{c}{(3600\,A)}                                                                   \\
            \cmidrule(lr){3-4} \cmidrule(lr){5-6} \cmidrule(lr){7-8} \cmidrule(lr){9-10} \cmidrule(lr){11-12} \cmidrule(lr){13-14} \cmidrule(lr){15-16}
            {Geom.}  & {Ph.} & {[\%]}                       & {[A]}                        & {[\%]}                        & {[A]}                         & {[\%]}                        & {[A]}                         & {[\%]}                        & {[A]}  & {[\%]}  & {[A]}  & {[\%]}  & {[A]}  & {[\%]}  & {[A]}   \\
            \midrule
            Parallel & L1    & 0.084                        & 0.13                         & -0.132                        & 0.79                          & -0.218                        & 3.27                          & -0.276                        & 6.62   & -0.318  & 8.58   & -0.395  & 11.86  & -1.857  & 66.85   \\
            Parallel & L2    & 0.169                        & 0.25                         & -0.039                        & 0.23                          & -0.131                        & 1.96                          & -0.598                        & 14.34  & -1.415  & 38.20  & -2.537  & 76.12  & -5.113  & 184.08  \\
            Parallel & L3    & 0.300                        & 0.45                         & 0.045                         & 0.27                          & -0.046                        & 0.69                          & -0.086                        & 2.07   & -0.099  & 2.67   & -0.121  & 3.63   & -0.428  & 15.42   \\
            \bottomrule
        \end{tabular}
    }
\end{table}

% ===========================================================================
% NACHSPANN (Eingerückt & ohne Seitenzahlen im Inhaltsverzeichnis)
% ===========================================================================

% --- LITERATURVERZEICHNIS ---
\clearpage
% Manueller Eintrag ins Inhaltsverzeichnis:
% 1. \protect\numberline{} sorgt für die Einrückung (Platzhalter für Nummer)
% 2. Das leere Klammernpaar {} am Ende unterdrückt die Seitenzahl
\addtocontents{toc}{\protect\contentsline{section}{\protect\numberline{}Literaturverzeichnis}{}{}}

% Literaturverzeichnis ausgeben (ohne automatische Überschrift, da wir sie manuell setzen)
\section*{Literaturverzeichnis}
\printbibliography[heading=none]

% --- EIGENSTÄNDIGKEITSERKLÄRUNG ---
\clearpage
% Manueller Eintrag ins Inhaltsverzeichnis (Eingerückt, ohne Seitenzahl)
\addtocontents{toc}{\protect\contentsline{section}{\protect\numberline{}Eigenständigkeitserklärung}{}{}}

% Datei einbinden (Achtung: Du musst Schritt 2 beachten!)
\input{02_Inhalt/03_Nachspann/eigenstaendigkeitserklaerung.tex}

\end{document}